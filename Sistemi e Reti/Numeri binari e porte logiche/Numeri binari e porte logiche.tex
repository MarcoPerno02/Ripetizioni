\documentclass[12pt,a4paper]{article}
\usepackage[utf8]{inputenc}
\usepackage{amsmath, amssymb}
\usepackage{graphicx}
\usepackage{geometry}
\usepackage{hyperref}
\usepackage{longtable}
\geometry{margin=2.5cm}

\title{Numeri e Logica Digitale: Teoria ed Esercizi}
\author{Corso di Sistemi e Reti}
\date{}

\begin{document}

\maketitle

\tableofcontents
\newpage

\section{Numeri in base 2 (binario)}

Il sistema binario è la base dei calcolatori digitali, in quanto ogni bit può rappresentare solo due stati: 0 e 1.

\subsection{Rappresentazione dei numeri binari}

Ogni cifra in binario è detta \textbf{bit}. Ad esempio:
\[
1011_2 = 1 \cdot 2^3 + 0 \cdot 2^2 + 1 \cdot 2^1 + 1 \cdot 2^0 = 11_{10}
\]
Il pedice $_2$ indica che il numero è in base 2, mentre $_{10}$ indica la base decimale.
$2^n$ rappresenta il peso di ogni bit, partendo da destra con $2^0$.
Ricorda che:
\[
\begin{aligned}
2^0 &= 1 \\
2^1 &= 2 \\
2^2 &= 4 \\
2^3 &= 8 \\
2^4 &= 16 \\
2^5 &= 32 \\
2^6 &= 64 \\
2^7 &= 128 \\
2^8 &= 256 \\
2^9 &= 512 \\
2^{10} &= 1024 = 1\text{K}  (1 Kilo) \\
2^{20} &= 1048576 = 1\text{M}  (1 Mega) \\
2^{30} &= 1073741824 = 1\text{G}  (1 Giga)
\end{aligned}
\]

\subsection{Conversione tra decimale e binario}

\textbf{Metodo della divisione per 2:}  
Per convertire un numero decimale in binario, si divide per 2 ripetutamente e si prendono i resti dal basso verso l'alto.

\textbf{Esempio:} Convertire $25_{10}$ in binario.
\begin{align*}
25 \div 2 &= 12 \text{ resto } 1 \\
12 \div 2 &= 6 \text{ resto } 0 \\
6 \div 2  &= 3 \text{ resto } 0 \\
3 \div 2  &= 1 \text{ resto } 1 \\
1 \div 2  &= 0 \text{ resto } 1
\end{align*}
Risultato: $25_{10} = 11001_2$

\subsection{Conversione tra binario e decimale}
\textbf{Metodo della somma ponderata:}
Per convertire un numero binario in decimale, si sommano i pesi dei bit che sono 1.
\textbf{Esempio:} Convertire $1101_2$ in decimale.
\[1101_2 = 1 \cdot 2^3 + 1 \cdot 2^2 + 0 \cdot 2^1 + 1 \cdot 2^0 = 8 + 4 + 0 + 1 = 13_{10}\]

\subsection{Esercizi svolti}
\begin{enumerate}
\item Convertire $43_{10}$ in binario:
\begin{align*}
43 \div 2 &= 21 \text{ resto } 1 \\
21 \div 2 &= 10 \text{ resto } 1 \\
10 \div 2 &= 5 \text{ resto } 0 \\
5 \div 2 &= 2 \text{ resto } 1 \\
2 \div 2 &= 1 \text{ resto } 0 \\
1 \div 2 &= 0 \text{ resto } 1
\end{align*}
Risultato: $43_{10} = 101011_2$

\item Convertire $14_{10}$ in binario:
\begin{align*}
14 \div 2 &= 7 \text{ resto } 0 \\
7 \div 2 &= 3 \text{ resto } 1 \\
3 \div 2 &= 1 \text{ resto } 1 \\
1 \div 2 &= 0 \text{ resto } 1
\end{align*}
Risultato: $14_{10} = 1110_2$

\item Convertire $101011_2$ in decimale:
\begin{align*}
101011_2 &= 1 \cdot 2^5 + 0 \cdot 2^4 + 1 \cdot 2^3 + 0 \cdot 2^2 + 1 \cdot 2^1 + 1 \cdot 2^0 \\ 
&= 32 + 0 + 8 + 0 + 2 + 1 = 43_{10}
\end{align*}

\item Convertire $1110_2$ in decimale:
\begin{align*}
1110_2 &= 1 \cdot 2^3 + 1 \cdot 2^2 + 1 \cdot 2^1 + 0 \cdot 2^0 \\
&= 8 + 4 + 2 + 0 = 14_{10}
\end{align*}
\end{enumerate}

\subsection{Esercizi da fare (finchè non si ha la nausea \texorpdfstring{:)}{(emoji felice)} )}
\begin{enumerate}
    \item Convertire $7_{10}$ in binario (111).
    \item Convertire $5_{10}$ in binario (101).
    \item Convertire $12_{10}$ in binario (1100).
    \item Convertire $78_{10}$ in binario (1001110).
    \item Convertire $255_{10}$ in binario (11111111).
    \item Convertire $100_{10}$ in binario (1100100).
    \item Convertire $200_{10}$ in binario (11001000).
    \item Convertire $120_{10}$ in binario (1111000).
    \item Convertire $111_2$ in decimale (7).
    \item Convertire $101_2$ in decimale (5).
    \item Convertire $1100_2$ in decimale (12).
    \item Convertire $1001110_2$ in decimale (78).
    \item Convertire $11111111_2$ in decimale (255).
    \item Convertire $1100100_2$ in decimale (100).
    \item Convertire $11001000_2$ in decimale (200).
    \item Convertire $1111000_2$ in decimale (120).
\end{enumerate}

\newpage
\section{Operazioni in binario}

\subsection{Addizione binaria}

Le regole fondamentali sono:
\[
0+0=0, \quad 0+1=1, \quad 1+0=1, \quad 1+1=10
\]
Quando si somma $1+1$, si scrive 0 e si riporta 1 alla colonna successiva.
Quando si somma $1+1+1$ (incluso il riporto), si scrive 1 e si riporta 1 alla colonna successiva.

\textbf{Esempio:} $1011_2 + 1101_2$
\[
\begin{array}{c c c c c}
 & 1 & 0 & 1 & 1 \\
+ & 1 & 1 & 0 & 1 \\
\hline
1 & 1 & 0 & 0 & 0
\end{array}
\]
Risultato: $11000_2$

\textbf{Esempio:} $111_2 + 101_2$
\[
\begin{array}{c c c c}
    & 1 & 1 & 1 \\
+   & 1 & 0 & 1 \\
\hline
1 & 1 & 0 & 0
\end{array}
\]
Risultato: $1100_2$

\textbf{Esempio:} $1111_2 + 1111_2$
\[
\begin{array}{c c c c c}
    & 1 & 1 & 1 & 1 \\
+   & 1 & 1 & 1 & 1 \\
\hline
1 & 1 & 1 & 1 & 0
\end{array}
\]
Risultato: $11110_2$
Nota come la somma di due 1 dove c'è un riporto produce un ulteriore riporto.

\subsection{Sottrazione binaria}

Si usa spesso il \textbf{complemento a 2}:
\begin{enumerate}
    \item Invertire i bit del sottraendo (complemento a 1)
    \item Sommare 1
    \item Sommare il risultato al minuendo
\end{enumerate}

\textbf{Esempio:} $1011_2 - 0101_2$

\[
0101_2 \rightarrow \text{complemento a 1} = 1010_2
\]

\[
1010_2 + 1 = 1011_2 \text{ (complemento a 2)}
\]

\[
1011_2 + 1011_2 = 10110_2 \text{ (scartare il riporto)}
\]
Risultato: $0110_2 = 6_{10}$

\subsection{Esercizi svolti}
\begin{enumerate}
    \item $1101_2 + 1011_2$
    \[\begin{array}{c c c c c}
     & 1 & 1 & 0 & 1 \\
    + & 1 & 0 & 1 & 1 \\
    \hline
    1 & 1 & 0 & 0 & 0
    \end{array}\]
    Risultato: $11000_2$

    \item $1001_2 - 0110_2$
    \[
    0110_2 \rightarrow \text{complemento a 1} = 1001_2
    \]
    \[
    1001_2 + 1 = 1010_2 \text{ (complemento a 2)}
    \]
    \[
    1001_2 + 1010_2 = 10011_2 \text{ (scartare il riporto)}
    \]
    \[
    \begin{array}{c c c c c}
        & 1 & 0 & 0 & 1 \\
    +   & 1 & 0 & 1 & 0 \\
    \hline
    1 & 0 & 0 & 1 & 1
    \end{array}
    \]
    Risultato: $0011_2 = 3_{10}$
\end{enumerate}

\subsection{Esercizi da fare}
\begin{enumerate}
    \item $1010_2 + 1101_2$ (risultato: $10111_2$)
    \item $1110_2 + 1011_2$ (risultato: $11001_2$)
    \item $10000_2 + 1111_2$ (risultato: $11111_2$)
    \item $1101_2 - 0011_2$ (risultato: $0110_2$)
    \item $1010_2 - 0101_2$ (risultato: $101_2$)
    \item $1111_2 - 0110_2$ (risultato: $1001_2$)
\end{enumerate}

\newpage
\section{Numeri esadecimali}

\subsection{Concetti base}

Il sistema esadecimale ha 16 simboli: $0,1,2,\dots,9,A,B,C,D,E,F$.  
Ogni cifra esadecimale rappresenta 4 bit.

Le lettere $A$--$F$ rappresentano i valori decimali $10$--$15$, cioè:
$$A_{16}=10_{10},\; B_{16}=11_{10},\; C_{16}=12_{10},\; D_{16}=13_{10},\; E_{16}=14_{10},\; F_{16}=15_{10}.$$
In binario:
$$A=1010_2,\; B=1011_2,\; C=1100_2,\; D=1101_2,\; E=1110_2,\; F=1111_2.$$

\subsection{Conversione binario \texorpdfstring{$\leftrightarrow$}{<->} esadecimale}

\textbf{Esempio:} $11011101_2$  
Dividere in gruppi di 4 bit: $1101\ 1101$  
$1101_2 = D_{16}$, $1101_2 = D_{16}$  
Risultato: $DD_{16}$

\textbf{Esempio:} $10100011_2$
Dividere in gruppi di 4 bit: $1010\ 0011$  
$1010_2 = A_{16}$, $0011_2 = 3_{16}$  
Risultato: $A3_{16}$

\subsection{Conversione esadecimale \texorpdfstring{$\to$}{->} decimale}

\[
2F_{16} = 2\cdot16^1 + 15\cdot16^0 = 32 + 15 = 47_{10}
\]

\subsection{Convertire decimale in esadecimale}
\textbf{Metodo della divisione per 16:}
Per convertire un numero decimale in esadecimale, si divide
per 16 ripetutamente e si prendono i resti dal basso verso l'alto.
\textbf{Esempio:} Convertire $254_{10}$ in esadecimale.
\begin{align*}
254 \div 16 &= 15 \text{ resto } 14 (E) \\
15 \div 16 &= 0 \text{ resto } 15 (F)
\end{align*}
Risultato: $254_{10} = FE_{16}$

\subsection{Esercizi svolti}
\begin{enumerate}
    \item Convertire $3B_{16}$ in binario:
    \[
    3 = 0011, \quad B = 1011 \\
    \Rightarrow 3B_{16} = 00111011_2
    \]

    \item Convertire $A7_{16}$ in binario:
    \[
    A = 1010, \quad 7 = 0111 \\
    \Rightarrow A7_{16} = 10100111_2
    \]

    \item Convertire $7E_{16}$ in decimale:
    \[
    7E_{16} = 7\cdot16^1 + 14\cdot16^0 = 112 + 14 = 126_{10}
    \]

    \item Convertire $1F_{16}$ in decimale:
    \[
    1F_{16} = 1\cdot16^1 + 15\cdot16^0 = 16 + 15 = 31_{10}
    \]

    \item Convertire $10110101_2$ in esadecimale:
    \[
    1011\ 0101 \rightarrow B5_{16}
    \]

    \item Convertire $11110000_2$ in esadecimale:
    \[
    1111\ 0000 \rightarrow F0_{16}
    \]

    \item Convertire $255_{10}$ in esadecimale:
    \begin{align*}
    255 \div 16 &= 15 \text{ resto } 15 (F) \\
    15 \div 16 &= 0 \text{ resto } 15 (F)
    \end{align*}
    Risultato: $FF_{16}$

    \item Convertire $100_{10}$ in esadecimale:
    \begin{align*}
    100 \div 16 &= 6 \text{ resto } 4 \\
    6 \div 16 &= 0 \text{ resto } 6
    \end{align*}
    Risultato: $64_{16}$

\end{enumerate}


\newpage
\section{Floating Point (virgola mobile)}

\subsection{Concetti base}

I numeri reali si rappresentano in forma normalizzata:
\[
N = (-1)^S \cdot 1.M \cdot 2^E
\]

\begin{itemize}
    \item $S$ = segno (0=positivo, 1=negativo)
    \item $M$ = mantissa (parte significativa)
    \item $E$ = esponente
\end{itemize}

\subsection{Standard IEEE 754 (32 bit)}

\begin{itemize}
    \item 1 bit per il segno
    \item 8 bit per l’esponente
    \item 23 bit per la mantissa
\end{itemize}

\subsection{Esempio}

Rappresentare $-5.25$ in floating point:

\[
5.25_{10} = 101.01_2 = 1.0101_2 \times 2^2
\]
\[
S=1, \quad E=2+127=129=10000001_2, \quad M=010100\ldots0
\]

\newpage
\section{Porte logiche}

Le porte logiche sono blocchi base dei circuiti digitali. Elaborano segnali 0 e 1.

\subsection{Porte fondamentali}

\begin{longtable}{|c|c|c|}
\hline
Porta & Simbolo & Funzione \\
\hline
AND & $\land$ & 1 se tutti ingressi = 1 \\
OR & $\lor$ & 1 se almeno un ingresso = 1 \\
NOT & $\lnot$ & inversione 0 $\leftrightarrow$ 1 \\
NAND & $\uparrow$ & negazione AND \\
NOR & $\downarrow$ & negazione OR \\
XOR & $\oplus$ & 1 se ingressi diversi \\
\hline
\end{longtable}

\subsection{Tabelle di verità}

\subsubsection{AND}
\begin{tabular}{c c | c}
A & B & A AND B \\
\hline
0 & 0 & 0 \\
0 & 1 & 0 \\
1 & 0 & 0 \\
1 & 1 & 1 \\
\end{tabular}

\subsubsection{OR}
\begin{tabular}{c c | c}
A & B & A OR B \\
\hline
0 & 0 & 0 \\
0 & 1 & 1 \\
1 & 0 & 1 \\
1 & 1 & 1 \\
\end{tabular}

\subsubsection{NOT}
\begin{tabular}{c | c}
A & NOT A \\
\hline
0 & 1 \\
1 & 0 \\
\end{tabular}

\subsubsection{XOR}
\begin{tabular}{c c | c}
A & B & A XOR B \\
\hline
0 & 0 & 0 \\
0 & 1 & 1 \\
1 & 0 & 1 \\
1 & 1 & 0 \\
\end{tabular}

\subsection{Esercizi svolti}

\begin{enumerate}
    \item Costruire la tabella di verità della porta NAND a due ingressi:
    \[
    A\quad B\quad A\text{ NAND }B
    \]
    Soluzione:
    \[
    0\ 0\ 1, \quad 0\ 1\ 1, \quad 1\ 0\ 1, \quad 1\ 1\ 0
    \]

    \item Costruire la tabella di verità della porta NOR a due ingressi:
    \[
    A\quad B\quad A\text{ NOR }B
    \]
    Soluzione:
    \[
    0\ 0\ 1, \quad 0\ 1\ 0, \quad 1\ 0\ 0, \quad 1\ 1\ 0
    \]
\end{enumerate}

\subsection{Esercizi avanzati: Circuiti combinatori}

\begin{enumerate}
    \item \textbf{Dato il circuito, trovare la tabella di verità:}
    
    Circuito: $(A \land B) \lor (\lnot C)$
    
    \begin{center}
    \begin{tabular}{|c|c|c||c|c|c|}
    \hline
    A & B & C & $A \land B$ & $\lnot C$ & Uscita \\
    \hline
    0 & 0 & 0 & 0 & 1 & 1 \\
    0 & 0 & 1 & 0 & 0 & 0 \\
    0 & 1 & 0 & 0 & 1 & 1 \\
    0 & 1 & 1 & 0 & 0 & 0 \\
    1 & 0 & 0 & 0 & 1 & 1 \\
    1 & 0 & 1 & 0 & 0 & 0 \\
    1 & 1 & 0 & 1 & 1 & 1 \\
    1 & 1 & 1 & 1 & 0 & 1 \\
    \hline
    \end{tabular}
    \end{center}

    \item \textbf{Dato il circuito, trovare la tabella di verità:}
    
    Circuito: $(A \oplus B) \land C$
    
    \begin{center}
    \begin{tabular}{|c|c|c||c|c|}
    \hline
    A & B & C & $A \oplus B$ & Uscita \\
    \hline
    0 & 0 & 0 & 0 & 0 \\
    0 & 0 & 1 & 0 & 0 \\
    0 & 1 & 0 & 1 & 0 \\
    0 & 1 & 1 & 1 & 1 \\
    1 & 0 & 0 & 1 & 0 \\
    1 & 0 & 1 & 1 & 1 \\
    1 & 1 & 0 & 0 & 0 \\
    1 & 1 & 1 & 0 & 0 \\
    \hline
    \end{tabular}
    \end{center}

    \item \textbf{Dato il circuito, trovare la tabella di verità:}
    
    Circuito: $\lnot(A \land B) \lor (B \land C)$
    
    \begin{center}
    \begin{tabular}{|c|c|c||c|c|c|c|}
    \hline
    A & B & C & $A \land B$ & $\lnot(A \land B)$ & $B \land C$ & Uscita \\
    \hline
    0 & 0 & 0 & 0 & 1 & 0 & 1 \\
    0 & 0 & 1 & 0 & 1 & 0 & 1 \\
    0 & 1 & 0 & 0 & 1 & 0 & 1 \\
    0 & 1 & 1 & 0 & 1 & 1 & 1 \\
    1 & 0 & 0 & 0 & 1 & 0 & 1 \\
    1 & 0 & 1 & 0 & 1 & 0 & 1 \\
    1 & 1 & 0 & 1 & 0 & 0 & 0 \\
    1 & 1 & 1 & 1 & 0 & 1 & 1 \\
    \hline
    \end{tabular}
    \end{center}

    \item \textbf{Data la tabella di verità, trovare il circuito logico:}
    
    \begin{center}
    \begin{tabular}{|c|c|c|}
    \hline
    A & B & Uscita \\
    \hline
    0 & 0 & 1 \\
    0 & 1 & 0 \\
    1 & 0 & 0 \\
    1 & 1 & 0 \\
    \hline
    \end{tabular}
    \end{center}
    
    Soluzione: $\lnot A \land \lnot B$ oppure $\lnot(A \lor B)$ (porta NOR)

    \item \textbf{Data la tabella di verità, trovare il circuito logico:}
    
    \begin{center}
    \begin{tabular}{|c|c|c|c|}
    \hline
    A & B & C & Uscita \\
    \hline
    0 & 0 & 0 & 0 \\
    0 & 0 & 1 & 1 \\
    0 & 1 & 0 & 1 \\
    0 & 1 & 1 & 1 \\
    1 & 0 & 0 & 1 \\
    1 & 0 & 1 & 1 \\
    1 & 1 & 0 & 1 \\
    1 & 1 & 1 & 1 \\
    \hline
    \end{tabular}
    \end{center}
    
    Soluzione: $A \lor B \lor C$ (L'uscita è 1 se almeno uno degli ingressi è 1)

    \item \textbf{Dato il circuito, trovare la tabella di verità:}
    
    Circuito: $(A \land \lnot B) \lor (\lnot A \land B)$
    
    Nota: questa è equivalente a $A \oplus B$
    
    \begin{center}
    \begin{tabular}{|c|c||c|c|c|c|c|}
    \hline
    A & B & $\lnot A$ & $\lnot B$ & $A \land \lnot B$ & $\lnot A \land B$ & Uscita \\
    \hline
    0 & 0 & 1 & 1 & 0 & 0 & 0 \\
    0 & 1 & 1 & 0 & 0 & 1 & 1 \\
    1 & 0 & 0 & 1 & 1 & 0 & 1 \\
    1 & 1 & 0 & 0 & 0 & 0 & 0 \\
    \hline
    \end{tabular}
    \end{center}

\end{enumerate}

\subsection{Esercizi da fare}

\begin{enumerate}
    \item Costruire la tabella di verità per: $(A \lor B) \land \lnot C$
    
    \textbf{Soluzione:}
    \begin{center}
    \begin{tabular}{|c|c|c||c|c|c|}
    \hline
    A & B & C & $A \lor B$ & $\lnot C$ & Uscita \\
    \hline
    0 & 0 & 0 & 0 & 1 & 0 \\
    0 & 0 & 1 & 0 & 0 & 0 \\
    0 & 1 & 0 & 1 & 1 & 1 \\
    0 & 1 & 1 & 1 & 0 & 0 \\
    1 & 0 & 0 & 1 & 1 & 1 \\
    1 & 0 & 1 & 1 & 0 & 0 \\
    1 & 1 & 0 & 1 & 1 & 1 \\
    1 & 1 & 1 & 1 & 0 & 0 \\
    \hline
    \end{tabular}
    \end{center}

    \item Costruire la tabella di verità per: $\lnot(A \oplus B)$ (XNOR)
    
    \textbf{Soluzione:}
    \begin{center}
    \begin{tabular}{|c|c||c|c|}
    \hline
    A & B & $A \oplus B$ & Uscita \\
    \hline
    0 & 0 & 0 & 1 \\
    0 & 1 & 1 & 0 \\
    1 & 0 & 1 & 0 \\
    1 & 1 & 0 & 1 \\
    \hline
    \end{tabular}
    \end{center}

    \item Dato il circuito $\lnot A \lor (B \land C)$, costruire la tabella di verità completa
    
    \textbf{Soluzione:}
    \begin{center}
    \begin{tabular}{|c|c|c||c|c|c|}
    \hline
    A & B & C & $\lnot A$ & $B \land C$ & Uscita \\
    \hline
    0 & 0 & 0 & 1 & 0 & 1 \\
    0 & 0 & 1 & 1 & 0 & 1 \\
    0 & 1 & 0 & 1 & 0 & 1 \\
    0 & 1 & 1 & 1 & 1 & 1 \\
    1 & 0 & 0 & 0 & 0 & 0 \\
    1 & 0 & 1 & 0 & 0 & 0 \\
    1 & 1 & 0 & 0 & 0 & 0 \\
    1 & 1 & 1 & 0 & 1 & 1 \\
    \hline
    \end{tabular}
    \end{center}

    \item Data la tabella: A=0,B=0 $\to$ 1; A=0,B=1 $\to$ 1; A=1,B=0 $\to$ 1; A=1,B=1 $\to$ 0. Trovare il circuito.
    
    \textbf{Soluzione:} $\lnot(A \land B)$ oppure $A \text{ NAND } B$

    \item Costruire la tabella di verità per: $(A \land B) \oplus C$
    
    \textbf{Soluzione:}
    \begin{center}
    \begin{tabular}{|c|c|c||c|c|}
    \hline
    A & B & C & $A \land B$ & Uscita \\
    \hline
    0 & 0 & 0 & 0 & 0 \\
    0 & 0 & 1 & 0 & 1 \\
    0 & 1 & 0 & 0 & 0 \\
    0 & 1 & 1 & 0 & 1 \\
    1 & 0 & 0 & 0 & 0 \\
    1 & 0 & 1 & 0 & 1 \\
    1 & 1 & 0 & 1 & 1 \\
    1 & 1 & 1 & 1 & 0 \\
    \hline
    \end{tabular}
    \end{center}

    \item Trovare il circuito per l'uscita che vale 1 solo quando esattamente due ingressi su A,B,C sono 1
    
    \textbf{Soluzione:} $(A \land B \land \lnot C) \lor (A \land \lnot B \land C) \lor (\lnot A \land B \land C)$
    
    \begin{center}
    \begin{tabular}{|c|c|c||c|}
    \hline
    A & B & C & Uscita \\
    \hline
    0 & 0 & 0 & 0 \\
    0 & 0 & 1 & 0 \\
    0 & 1 & 0 & 0 \\
    0 & 1 & 1 & 1 \\
    1 & 0 & 0 & 0 \\
    1 & 0 & 1 & 1 \\
    1 & 1 & 0 & 1 \\
    1 & 1 & 1 & 0 \\
    \hline
    \end{tabular}
    \end{center}

\end{enumerate}

\end{document}
