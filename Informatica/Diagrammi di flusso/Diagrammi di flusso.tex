\documentclass[a4paper,16pt]{article}

% Pacchetti necessari
\usepackage[utf8]{inputenc}
\usepackage[italian]{babel}
\usepackage{amsmath}
\usepackage{amssymb}
\usepackage{geometry}
\usepackage{tikz}
\usetikzlibrary{shapes.geometric, arrows, positioning, calc}
\usepackage{xcolor}
\usepackage{listings}
\usepackage{fancyhdr}
\usepackage{tcolorbox}
\usepackage{enumitem}

% Configurazione della pagina
\geometry{margin=2.5cm}
\pagestyle{fancy}
\fancyhf{}
\rhead{Diagrammi di Flusso}
\lhead{Guida Completa}
\cfoot{\thepage}

% Stili per i diagrammi di flusso
\tikzstyle{startstop} = [rectangle, rounded corners, minimum width=3cm, minimum height=1cm, text centered, draw=black, fill=red!30]
\tikzstyle{process} = [rectangle, minimum width=3cm, minimum height=1cm, text centered, draw=black, fill=orange!30]
\tikzstyle{decision} = [diamond, minimum width=3cm, minimum height=1cm, text centered, draw=black, fill=green!30, aspect=2]
\tikzstyle{io} = [trapezium, trapezium left angle=70, trapezium right angle=110, minimum width=3cm, minimum height=1cm, text centered, draw=black, fill=blue!30]
\tikzstyle{arrow} = [thick,->,>=stealth]

% Configurazione per il codice
\lstset{
    basicstyle=\ttfamily\small,
    breaklines=true,
    frame=single,
    numbers=left,
    numberstyle=\tiny,
    backgroundcolor=\color{gray!10}
}

\title{\textbf{Diagrammi di Flusso}\\[0.3cm]\large Guida Completa con Esercizi Svolti}
\author{Materiale Didattico}
\date{\today}

\begin{document}

\maketitle
\tableofcontents
\newpage

\section{Introduzione ai Diagrammi di Flusso}

\subsection{Cosa sono i Diagrammi di Flusso}

I \textbf{diagrammi di flusso} (o \textit{flowchart}) sono rappresentazioni grafiche di algoritmi o processi. Utilizzano simboli geometrici standardizzati collegati da frecce per mostrare la sequenza di operazioni da eseguire per risolvere un problema o completare un'attività.

\subsubsection{Perché sono importanti?}

I diagrammi di flusso sono fondamentali nella programmazione per diversi motivi:

\begin{itemize}[leftmargin=*]
    \item \textbf{Visualizzazione}: Permettono di vedere graficamente la logica di un algoritmo
    \item \textbf{Progettazione}: Aiutano a pianificare il codice prima di scriverlo
    \item \textbf{Comunicazione}: Facilitano la condivisione delle idee con altri programmatori
    \item \textbf{Debug}: Rendono più semplice individuare errori logici
    \item \textbf{Documentazione}: Forniscono documentazione visiva del codice
\end{itemize}

\subsection{Simboli Standard}

Ogni forma geometrica in un diagramma di flusso ha un significato specifico:

\begin{center}
\begin{tikzpicture}[node distance=2.5cm]

% Inizio/Fine
\node (start) [startstop] {Inizio/Fine};
\node [right=1cm of start, text width=5cm, align=left] {Indica l'inizio o la fine del programma. Forma: rettangolo con angoli arrotondati};

% Processo
\node (process) [process, below=0.8cm of start] {Operazione};
\node [right=1cm of process, text width=5cm, align=left] {Rappresenta un'operazione o un'istruzione da eseguire. Forma: rettangolo};

% Input/Output
\node (io) [io, below=0.8cm of process] {Input/Output};
\node [right=1cm of io, text width=5cm, align=left] {Indica operazioni di lettura (input) o scrittura (output). Forma: parallelogramma};

% Decisione
\node (decision) [decision, below=1.2cm of io] {Decisione?};
\node [right=1cm of decision, text width=5cm, align=left] {Rappresenta una condizione che richiede una scelta. Forma: rombo};

% Frecce
\node (arrow) [below=0.8cm of decision] {};
\draw [arrow] ([xshift=-1cm]arrow.north) -- ([xshift=-1cm]arrow.south);
\node [right=0cm of arrow, text width=5cm, align=left] {Indica il flusso di esecuzione e la direzione delle operazioni};

\end{tikzpicture}
\end{center}

\subsection{Regole di Base}

Quando si disegnano diagrammi di flusso, è importante seguire alcune regole:

\begin{enumerate}[leftmargin=*]
    \item \textbf{Un solo inizio e una sola fine}: Ogni diagramma deve avere esattamente un punto di partenza e un punto di arrivo
    \item \textbf{Flusso dall'alto verso il basso}: Generalmente il flusso procede dall'alto verso il basso e da sinistra a destra
    \item \textbf{Frecce chiare}: Le frecce devono indicare chiaramente la direzione del flusso
    \item \textbf{Nessun incrocio di linee}: Evitare che le linee si incrocino (quando possibile)
    \item \textbf{Chiarezza}: Ogni blocco deve contenere istruzioni semplici e comprensibili
    \item \textbf{Decisioni binarie}: I rombi devono avere esattamente due uscite (Vero/Falso, Sì/No)
\end{enumerate}

\newpage
\section{Le Variabili}

\subsection{Cosa sono le Variabili}

Una \textbf{variabile} è un contenitore che memorizza un valore nella memoria del computer. Ogni variabile ha:

\begin{itemize}[leftmargin=*]
    \item \textbf{Nome}: Un identificatore univoco (es. \texttt{x}, \texttt{eta}, \texttt{somma})
    \item \textbf{Tipo}: Il tipo di dato che può contenere (es. numero intero, decimale, testo)
    \item \textbf{Valore}: Il contenuto attuale della variabile
\end{itemize}

\subsection{Operazioni con le Variabili}

Le operazioni principali che si possono fare con le variabili sono:

\begin{enumerate}[leftmargin=*]
    \item \textbf{Assegnazione}: Memorizzare un valore in una variabile\\
    Sintassi: \texttt{variabile $\leftarrow$ valore} o \texttt{variabile = valore}
    
    \item \textbf{Lettura (Input)}: Acquisire un valore dall'utente\\
    Sintassi: \texttt{Leggi variabile} o \texttt{Input variabile}
    
    \item \textbf{Scrittura (Output)}: Mostrare il valore di una variabile\\
    Sintassi: \texttt{Scrivi variabile} o \texttt{Output variabile}
    
    \item \textbf{Operazioni aritmetiche}: Calcoli matematici\\
    Esempi: \texttt{somma $\leftarrow$ a + b}, \texttt{prodotto $\leftarrow$ x * y}

    \item \textbf{Confronti}: Confronti tra altre variabili o valori dello stesso tipo\\
    Esempi: \texttt{a > b}, \texttt{età > 18}
\end{enumerate}

\subsection{Esempio Semplice: Assegnazione}

Vediamo come rappresentare l'assegnazione di un valore a una variabile:

\begin{center}
\begin{tikzpicture}[node distance=2cm]
\node (start) [startstop] {Inizio};
\node (assign) [process, below of=start] {$x \leftarrow 10$};
\node (output) [io, below of=assign] {Scrivi x};
\node (stop) [startstop, below of=output] {Fine};

\draw [arrow] (start) -- (assign);
\draw [arrow] (assign) -- (output);
\draw [arrow] (output) -- (stop);
\end{tikzpicture}
\end{center}

Questo diagramma esegue le seguenti operazioni:
\begin{enumerate}
    \item Inizia il programma
    \item Assegna il valore 10 alla variabile $x$
    \item Mostra il valore di $x$ (stampa 10)
    \item Termina il programma
\end{enumerate}

\newpage
\subsection{Esercizi Svolti con Variabili}

\subsubsection{Esercizio 1: Assegnazione e Output}

\begin{tcolorbox}[colback=blue!5!white,colframe=blue!75!black,title=Traccia]
Creare un diagramma di flusso che assegni il valore 25 alla variabile \texttt{eta} e poi lo visualizzi.
\end{tcolorbox}

\textbf{Soluzione:}

\begin{center}
\begin{tikzpicture}[node distance=2cm]
\node (start) [startstop] {Inizio};
\node (assign) [process, below of=start] {$eta \leftarrow 25$};
\node (output) [io, below of=assign] {Scrivi eta};
\node (stop) [startstop, below of=output] {Fine};

\draw [arrow] (start) -- (assign);
\draw [arrow] (assign) -- (output);
\draw [arrow] (output) -- (stop);
\end{tikzpicture}
\end{center}

\textbf{Spiegazione:}
\begin{enumerate}
    \item Il programma inizia
    \item Viene assegnato il valore 25 alla variabile \texttt{eta}
    \item Il valore di \texttt{eta} viene stampato (risultato: 25)
    \item Il programma termina
\end{enumerate}

\textbf{Codice equivalente (pseudocodice):}
\begin{lstlisting}
INIZIO
    eta <- 25
    SCRIVI eta
FINE
\end{lstlisting}

\newpage
\subsubsection{Esercizio 2: Input e Output}

\begin{tcolorbox}[colback=blue!5!white,colframe=blue!75!black,title=Traccia]
Creare un diagramma di flusso che chieda all'utente di inserire il proprio nome e poi lo saluti stampando "Ciao, [nome]!".
\end{tcolorbox}

\textbf{Soluzione:}

\begin{center}
\begin{tikzpicture}[node distance=2cm]
\node (start) [startstop] {Inizio};
\node (input) [io, below of=start] {Leggi nome};
\node (output) [io, below of=input] {Scrivi "Ciao, ", nome, "!"};
\node (stop) [startstop, below of=output] {Fine};

\draw [arrow] (start) -- (input);
\draw [arrow] (input) -- (output);
\draw [arrow] (output) -- (stop);
\end{tikzpicture}
\end{center}

\textbf{Spiegazione:}
\begin{enumerate}
    \item Il programma inizia
    \item L'utente inserisce il proprio nome che viene memorizzato nella variabile \texttt{nome}
    \item Viene stampato il messaggio di saluto personalizzato
    \item Il programma termina
\end{enumerate}

\textbf{Esempio di esecuzione:}
\begin{itemize}
    \item Input: \texttt{Mario}
    \item Output: \texttt{Ciao, Mario!}
\end{itemize}

\textbf{Codice equivalente (pseudocodice):}
\begin{lstlisting}
INIZIO
    LEGGI nome
    SCRIVI "Ciao, ", nome, "!"
FINE
\end{lstlisting}

\newpage
\subsubsection{Esercizio 3: Operazioni Aritmetiche Semplici}

\begin{tcolorbox}[colback=blue!5!white,colframe=blue!75!black,title=Traccia]
Creare un diagramma di flusso che calcoli il doppio di un numero. Il programma deve assegnare il valore 7 alla variabile \texttt{numero}, calcolare il doppio e visualizzare il risultato.
\end{tcolorbox}

\textbf{Soluzione:}

\begin{center}
\begin{tikzpicture}[node distance=2cm]
\node (start) [startstop] {Inizio};
\node (assign1) [process, below of=start] {$numero \leftarrow 7$};
\node (calc) [process, below of=assign1] {$doppio \leftarrow numero \times 2$};
\node (output) [io, below of=calc] {Scrivi doppio};
\node (stop) [startstop, below of=output] {Fine};

\draw [arrow] (start) -- (assign1);
\draw [arrow] (assign1) -- (calc);
\draw [arrow] (calc) -- (output);
\draw [arrow] (output) -- (stop);
\end{tikzpicture}
\end{center}

\textbf{Spiegazione:}
\begin{enumerate}
    \item Il programma inizia
    \item Viene assegnato il valore 7 alla variabile \texttt{numero}
    \item Viene calcolato il doppio: $doppio = 7 \times 2 = 14$
    \item Il valore di \texttt{doppio} viene stampato (risultato: 14)
    \item Il programma termina
\end{enumerate}

\textbf{Codice equivalente (pseudocodice):}
\begin{lstlisting}
INIZIO
    numero <- 7
    doppio <- numero * 2
    SCRIVI doppio
FINE
\end{lstlisting}

\newpage
\subsubsection{Esercizio 4: Somma di Due Numeri}

\begin{tcolorbox}[colback=blue!5!white,colframe=blue!75!black,title=Traccia]
Creare un diagramma di flusso che chieda all'utente due numeri, calcoli la loro somma e visualizzi il risultato.
\end{tcolorbox}

\textbf{Soluzione:}

\begin{center}
\begin{tikzpicture}[node distance=2cm]
\node (start) [startstop] {Inizio};
\node (input1) [io, below of=start] {Leggi a};
\node (input2) [io, below of=input1] {Leggi b};
\node (calc) [process, below of=input2] {$somma \leftarrow a + b$};
\node (output) [io, below of=calc] {Scrivi somma};
\node (stop) [startstop, below of=output] {Fine};

\draw [arrow] (start) -- (input1);
\draw [arrow] (input1) -- (input2);
\draw [arrow] (input2) -- (calc);
\draw [arrow] (calc) -- (output);
\draw [arrow] (output) -- (stop);
\end{tikzpicture}
\end{center}

\textbf{Spiegazione:}
\begin{enumerate}
    \item Il programma inizia
    \item L'utente inserisce il primo numero che viene memorizzato in \texttt{a}
    \item L'utente inserisce il secondo numero che viene memorizzato in \texttt{b}
    \item Viene calcolata la somma: $somma = a + b$
    \item Il risultato viene stampato
    \item Il programma termina
\end{enumerate}

\textbf{Esempio di esecuzione:}
\begin{itemize}
    \item Input 1: \texttt{a = 15}
    \item Input 2: \texttt{b = 23}
    \item Calcolo: $somma = 15 + 23 = 38$
    \item Output: \texttt{38}
\end{itemize}

\textbf{Codice equivalente (pseudocodice):}
\begin{lstlisting}
INIZIO
    LEGGI a
    LEGGI b
    somma <- a + b
    SCRIVI somma
FINE
\end{lstlisting}

\newpage
\subsubsection{Esercizio 5: Calcolo della Media}

\begin{tcolorbox}[colback=blue!5!white,colframe=blue!75!black,title=Traccia]
Creare un diagramma di flusso che calcoli la media di tre numeri inseriti dall'utente e visualizzi il risultato.
\end{tcolorbox}

\textbf{Soluzione:}

\begin{center}
\begin{tikzpicture}[node distance=1.8cm]
\node (start) [startstop] {Inizio};
\node (input1) [io, below of=start] {Leggi num1};
\node (input2) [io, below of=input1] {Leggi num2};
\node (input3) [io, below of=input2] {Leggi num3};
\node (calc1) [process, below of=input3] {$somma \leftarrow num1 + num2 + num3$};
\node (calc2) [process, below of=calc1] {$media \leftarrow somma / 3$};
\node (output) [io, below of=calc2] {Scrivi media};
\node (stop) [startstop, below of=output] {Fine};

\draw [arrow] (start) -- (input1);
\draw [arrow] (input1) -- (input2);
\draw [arrow] (input2) -- (input3);
\draw [arrow] (input3) -- (calc1);
\draw [arrow] (calc1) -- (calc2);
\draw [arrow] (calc2) -- (output);
\draw [arrow] (output) -- (stop);
\end{tikzpicture}
\end{center}

\textbf{Spiegazione:}
\begin{enumerate}
    \item Il programma inizia
    \item L'utente inserisce tre numeri: \texttt{num1}, \texttt{num2}, \texttt{num3}
    \item Viene calcolata la somma dei tre numeri
    \item Viene calcolata la media dividendo la somma per 3
    \item Il risultato viene stampato
    \item Il programma termina
\end{enumerate}

\textbf{Esempio di esecuzione:}
\begin{itemize}
    \item Input 1: \texttt{num1 = 8}
    \item Input 2: \texttt{num2 = 7}
    \item Input 3: \texttt{num3 = 9}
    \item Calcolo somma: $somma = 8 + 7 + 9 = 24$
    \item Calcolo media: $media = 24 / 3 = 8$
    \item Output: \texttt{8}
\end{itemize}

\textbf{Codice equivalente (pseudocodice):}
\begin{lstlisting}
INIZIO
    LEGGI num1
    LEGGI num2
    LEGGI num3
    somma <- num1 + num2 + num3
    media <- somma / 3
    SCRIVI media
FINE
\end{lstlisting}

\newpage
\subsubsection{Esercizio 6: Area di un Rettangolo}

\begin{tcolorbox}[colback=blue!5!white,colframe=blue!75!black,title=Traccia]
Creare un diagramma di flusso che calcoli l'area di un rettangolo. Il programma deve chiedere all'utente di inserire base e altezza, calcolare l'area e visualizzare il risultato.
\end{tcolorbox}

\textbf{Soluzione:}

\begin{center}
\begin{tikzpicture}[node distance=2cm]
\node (start) [startstop] {Inizio};
\node (input1) [io, below of=start] {Leggi base};
\node (input2) [io, below of=input1] {Leggi altezza};
\node (calc) [process, below of=input2] {$area \leftarrow base \times altezza$};
\node (output) [io, below of=calc] {Scrivi area};
\node (stop) [startstop, below of=output] {Fine};

\draw [arrow] (start) -- (input1);
\draw [arrow] (input1) -- (input2);
\draw [arrow] (input2) -- (calc);
\draw [arrow] (calc) -- (output);
\draw [arrow] (output) -- (stop);
\end{tikzpicture}
\end{center}

\textbf{Spiegazione:}
\begin{enumerate}
    \item Il programma inizia
    \item L'utente inserisce la base del rettangolo
    \item L'utente inserisce l'altezza del rettangolo
    \item Viene calcolata l'area: $area = base \times altezza$
    \item Il risultato viene stampato
    \item Il programma termina
\end{enumerate}

\textbf{Esempio di esecuzione:}
\begin{itemize}
    \item Input 1: \texttt{base = 5}
    \item Input 2: \texttt{altezza = 8}
    \item Calcolo: $area = 5 \times 8 = 40$
    \item Output: \texttt{L'area del rettangolo è: 40}
\end{itemize}

\textbf{Codice equivalente (pseudocodice):}
\begin{lstlisting}
INIZIO
    LEGGI base
    LEGGI altezza
    area <- base * altezza
    SCRIVI "L'area del rettangolo e': ", area
FINE
\end{lstlisting}

\newpage
\subsubsection{Esercizio 7: Conversione Temperature (Celsius - Fahrenheit)}

\begin{tcolorbox}[colback=blue!5!white,colframe=blue!75!black,title=Traccia]
Creare un diagramma di flusso che converta una temperatura da gradi Celsius a gradi Fahrenheit. Formula: $F = C \times 1.8 + 32$
\end{tcolorbox}

\textbf{Soluzione:}

\begin{center}
\begin{tikzpicture}[node distance=2cm]
\node (start) [startstop] {Inizio};
\node (input) [io, below of=start] {Leggi celsius};
\node (calc) [process, below of=input, text width=3.5cm] {$fahrenheit \leftarrow celsius \times 1.8 + 32$};
\node (output) [io, below of=calc] {Scrivi fahrenheit};
\node (stop) [startstop, below of=output] {Fine};

\draw [arrow] (start) -- (input);
\draw [arrow] (input) -- (calc);
\draw [arrow] (calc) -- (output);
\draw [arrow] (output) -- (stop);
\end{tikzpicture}
\end{center}

\textbf{Spiegazione:}
\begin{enumerate}
    \item Il programma inizia
    \item L'utente inserisce la temperatura in gradi Celsius
    \item Viene applicata la formula di conversione: $fahrenheit = celsius \times 1.8 + 32$
    \item Il risultato in gradi Fahrenheit viene stampato
    \item Il programma termina
\end{enumerate}

\textbf{Esempio di esecuzione:}
\begin{itemize}
    \item Input: \texttt{celsius = 25}
    \item Calcolo: $fahrenheit = 25 \times 1.8 + 32 = 45 + 32 = 77$
    \item Output: \texttt{77°F}
\end{itemize}

\textbf{Nota matematica:} La formula di conversione deriva dalla relazione lineare tra le due scale: il punto di congelamento dell'acqua è 0°C (32°F) e il punto di ebollizione è 100°C (212°F).

\textbf{Codice equivalente (pseudocodice):}
\begin{lstlisting}
INIZIO
    LEGGI celsius
    fahrenheit <- celsius * 1.8 + 32
    SCRIVI fahrenheit, " gradi Fahrenheit"
FINE
\end{lstlisting}

\newpage
\subsubsection{Esercizio 8: Scambio di Valori}

\begin{tcolorbox}[colback=blue!5!white,colframe=blue!75!black,title=Traccia]
Creare un diagramma di flusso che scambi i valori di due variabili \texttt{a} e \texttt{b}. Utilizzare una variabile temporanea \texttt{temp}.
\end{tcolorbox}

\textbf{Soluzione:}

\begin{center}
\begin{tikzpicture}[node distance=1.8cm]
\node (start) [startstop] {Inizio};
\node (input1) [io, below of=start] {Leggi a};
\node (input2) [io, below of=input1] {Leggi b};
\node (output1) [io, below of=input2, text width=3.5cm] {Scrivi "Prima: a=", a, " b=", b};
\node (swap1) [process, below of=output1] {$temp \leftarrow a$};
\node (swap2) [process, below of=swap1] {$a \leftarrow b$};
\node (swap3) [process, below of=swap2] {$b \leftarrow temp$};
\node (output2) [io, below of=swap3, text width=3.5cm] {Scrivi "Dopo: a=", a, " b=", b};
\node (stop) [startstop, below of=output2] {Fine};

\draw [arrow] (start) -- (input1);
\draw [arrow] (input1) -- (input2);
\draw [arrow] (input2) -- (output1);
\draw [arrow] (output1) -- (swap1);
\draw [arrow] (swap1) -- (swap2);
\draw [arrow] (swap2) -- (swap3);
\draw [arrow] (swap3) -- (output2);
\draw [arrow] (output2) -- (stop);
\end{tikzpicture}
\end{center}

\textbf{Spiegazione:}
\begin{enumerate}
    \item Il programma inizia
    \item L'utente inserisce i valori di \texttt{a} e \texttt{b}
    \item Vengono stampati i valori iniziali
    \item Il valore di \texttt{a} viene salvato nella variabile temporanea \texttt{temp}
    \item Il valore di \texttt{b} viene copiato in \texttt{a}
    \item Il valore di \texttt{temp} (che conteneva il vecchio valore di \texttt{a}) viene copiato in \texttt{b}
    \item Vengono stampati i valori dopo lo scambio
    \item Il programma termina
\end{enumerate}

\textbf{Esempio di esecuzione:}
\begin{itemize}
    \item Input 1: \texttt{a = 10}
    \item Input 2: \texttt{b = 20}
    \item Output iniziale: \texttt{Prima: a=10 b=20}
    \item Scambio:
    \begin{itemize}
        \item $temp = 10$ (salvo \texttt{a})
        \item $a = 20$ (copio \texttt{b} in \texttt{a})
        \item $b = 10$ (copio \texttt{temp} in \texttt{b})
    \end{itemize}
    \item Output finale: \texttt{Dopo: a=20 b=10}
\end{itemize}

\textbf{Perché serve la variabile temporanea?}\\
Senza la variabile \texttt{temp}, quando copiamo \texttt{b} in \texttt{a}, perdiamo il valore originale di \texttt{a} e non possiamo più assegnarlo a \texttt{b}.

\textbf{Codice equivalente (pseudocodice):}
\begin{lstlisting}
INIZIO
    LEGGI a
    LEGGI b
    SCRIVI "Prima: a=", a, " b=", b
    temp <- a
    a <- b
    b <- temp
    SCRIVI "Dopo: a=", a, " b=", b
FINE
\end{lstlisting}

\newpage
\section{Esercizi Proposti (Variabili)}

Per consolidare le conoscenze acquisite, prova a risolvere i seguenti esercizi creando i relativi diagrammi di flusso:

\begin{enumerate}[leftmargin=*]
    \item Creare un diagramma che calcoli il perimetro di un rettangolo (formula: $P = 2 \times (base + altezza)$)
    
    \item Creare un diagramma che calcoli l'area di un cerchio dato il raggio (formula: $A = \pi \times r^2$, usa $\pi \approx 3.14$)
    
    \item Creare un diagramma che converta una temperatura da Fahrenheit a Celsius (formula: $C = (F - 32) / 1.8$)
    
    \item Creare un diagramma che calcoli la media ponderata di due voti con i rispettivi pesi
    
    \item Creare un diagramma che calcoli il resto della divisione tra due numeri interi senza usare l'operatore modulo
    
    \item Creare un diagramma che calcoli l'IVA (22\%) su un prezzo e il prezzo finale
    
    \item Creare un diagramma che, dati i tre lati di un triangolo, calcoli il perimetro e l'area usando la formula di Erone
\end{enumerate}

\vspace{1cm}
\begin{tcolorbox}[colback=yellow!10!white,colframe=orange!75!black,title=Nota]
\textbf{Prossimo argomento:} Nella prossima sezione tratteremo i cicli (while, do-while, for) per eseguire operazioni ripetute.
\end{tcolorbox}

\newpage

\section{Blocchi Condizionali (Selezione)}

\subsection{Cosa sono i Blocchi Condizionali}

I \textbf{blocchi condizionali} (o strutture di selezione) permettono al programma di prendere decisioni e seguire percorsi diversi in base al valore di una condizione. Sono fondamentali per creare programmi che reagiscono in modo diverso a situazioni diverse.

\subsubsection{Concetto di Condizione}

Una \textbf{condizione} è un'espressione che può essere vera o falsa. Esempi:
\begin{itemize}[leftmargin=*]
    \item $x > 10$ (vero se x è maggiore di 10, altrimenti falso)
    \item $eta \geq 18$ (vero se eta è maggiore o uguale a 18)
    \item $nome = "Mario"$ (vero se nome è esattamente "Mario")
    \item $a \neq b$ (vero se a è diverso da b)
\end{itemize}

\subsection{Tipi di Strutture Condizionali}

\subsubsection{Selezione Semplice (if)}

Esegue un blocco di istruzioni solo se la condizione è vera.

\textbf{Sintassi:}
\begin{lstlisting}
SE condizione ALLORA
    istruzioni
FINE SE
\end{lstlisting}

\begin{center}
\begin{tikzpicture}[node distance=2.5cm]
\node (start) [startstop] {Inizio};
\node (input) [io, below of=start] {Leggi x};
\node (decision) [decision, below of=input] {$x > 0$?};
\node (process) [process, right=2cm of decision] {Scrivi "Positivo"};
\node (stop) [startstop, below=2cm of decision] {Fine};

\draw [arrow] (start) -- (input);
\draw [arrow] (input) -- (decision);
\draw [arrow] (decision) -- node[anchor=south] {Vero} (process);
\draw [arrow] (decision) -- node[anchor=east] {Falso} (stop);
\draw [arrow] (process) |- (stop);
\end{tikzpicture}
\end{center}

\newpage
\subsubsection{Selezione Binaria (if-else)}

Esegue un blocco se la condizione è vera, altrimenti esegue un blocco alternativo.

\textbf{Sintassi:}
\begin{lstlisting}
SE condizione ALLORA
    istruzioni_vero
ALTRIMENTI
    istruzioni_falso
FINE SE
\end{lstlisting}

\begin{center}
\begin{tikzpicture}[node distance=2cm]
\node (start) [startstop] {Inizio};
\node (input) [io, below of=start] {Leggi eta};
\node (decision) [decision, below of=input, yshift=-0.5cm] {$eta \geq 18$?};
\node (true) [process, left=2cm of decision, yshift=-2.5cm] {Scrivi "Maggiorenne"};
\node (false) [process, right=2cm of decision, yshift=-2.5cm] {Scrivi "Minorenne"};
\node (stop) [startstop, below=1cm of decision, yshift=-3cm] {Fine};

\draw [arrow] (start) -- (input);
\draw [arrow] (input) -- (decision);
\draw [arrow] (decision) -| node[anchor=south east] {Vero} (true);
\draw [arrow] (decision) -| node[anchor=south west] {Falso} (false);
\draw [arrow] (true) |- (stop);
\draw [arrow] (false) |- (stop);
\end{tikzpicture}
\end{center}

\subsubsection{Selezione Multipla (if-else if-else)}

Permette di verificare più condizioni in sequenza.

\textbf{Sintassi:}
\begin{lstlisting}
SE condizione1 ALLORA
    istruzioni1
ALTRIMENTI SE condizione2 ALLORA
    istruzioni2
ALTRIMENTI
    istruzioni3
FINE SE
\end{lstlisting}

\subsection{Operatori di Confronto}

\begin{center}
\begin{tabular}{|c|c|l|}
\hline
\textbf{Operatore} & \textbf{Significato} & \textbf{Esempio} \\
\hline
$=$ o $==$ & Uguale a & $x = 5$ \\
$\neq$ o $!=$ & Diverso da & $x \neq 0$ \\
$<$ & Minore di & $x < 10$ \\
$>$ & Maggiore di & $x > 0$ \\
$\leq$ & Minore o uguale & $x \leq 100$ \\
$\geq$ & Maggiore o uguale & $x \geq 18$ \\
\hline
\end{tabular}
\end{center}

\subsection{Operatori Logici}

Per combinare più condizioni:

\begin{center}
\begin{tabular}{|c|c|l|}
\hline
\textbf{Operatore} & \textbf{Significato} & \textbf{Esempio} \\
\hline
AND (E) & Entrambe vere & $(x > 0)$ AND $(x < 10)$ \\
OR (O) & Almeno una vera & $(x < 0)$ OR $(x > 100)$ \\
NOT (NON) & Negazione & NOT $(x = 0)$ \\
\hline
\end{tabular}
\end{center}

\newpage
\subsection{Esercizi Svolti con Blocchi Condizionali}

\subsubsection{Esercizio 1: Numero Positivo o Negativo}

\begin{tcolorbox}[colback=blue!5!white,colframe=blue!75!black,title=Traccia]
Creare un diagramma di flusso che legga un numero e stampi se è positivo, negativo o zero.
\end{tcolorbox}

\textbf{Soluzione:}

\begin{center}
\begin{tikzpicture}[node distance=2cm]
\node (start) [startstop] {Inizio};
\node (input) [io, below of=start] {Leggi numero};
\node (decision1) [decision, below of=input, yshift=-0.5cm] {$numero > 0$?};
\node (positive) [io, right=2.5cm of decision1, yshift=-1.5cm] {Scrivi "Positivo"};
\node (decision2) [decision, below=2cm of decision1] {$numero < 0$?};
\node (negative) [io, right=2.5cm of decision2, yshift=-1.5cm] {Scrivi "Negativo"};
\node (zero) [io, below=2cm of decision2] {Scrivi "Zero"};
\node (stop) [startstop, below=1cm of zero] {Fine};

\draw [arrow] (start) -- (input);
\draw [arrow] (input) -- (decision1);
\draw [arrow] (decision1) -| node[anchor=south east] {Vero} (positive);
\draw [arrow] (decision1) -- node[anchor=east] {Falso} (decision2);
\draw [arrow] (decision2) -| node[anchor=south east] {Vero} (negative);
\draw [arrow] (decision2) -- node[anchor=east] {Falso} (zero);
\draw [arrow] (positive) |- ([yshift=-0.5cm]positive.south) -| (stop);
\draw [arrow] (negative) |- ([yshift=-0.5cm]negative.south) -| (stop);
\draw [arrow] (zero) -- (stop);
\end{tikzpicture}
\end{center}

\textbf{Spiegazione:}
\begin{enumerate}
    \item Il programma legge un numero
    \item Prima condizione: verifica se il numero è maggiore di 0
    \begin{itemize}
        \item Se VERO: stampa "Positivo" e termina
        \item Se FALSO: passa alla seconda condizione
    \end{itemize}
    \item Seconda condizione: verifica se il numero è minore di 0
    \begin{itemize}
        \item Se VERO: stampa "Negativo" e termina
        \item Se FALSO: il numero deve essere zero, stampa "Zero"
    \end{itemize}
\end{enumerate}

\textbf{Esempi di esecuzione:}
\begin{itemize}
    \item Input: \texttt{5} → Output: \texttt{Positivo}
    \item Input: \texttt{-3} → Output: \texttt{Negativo}
    \item Input: \texttt{0} → Output: \texttt{Zero}
\end{itemize}

\textbf{Codice equivalente (pseudocodice):}
\begin{lstlisting}
INIZIO
    LEGGI numero
    SE numero > 0 ALLORA
        SCRIVI "Positivo"
    ALTRIMENTI SE numero < 0 ALLORA
        SCRIVI "Negativo"
    ALTRIMENTI
        SCRIVI "Zero"
    FINE SE
FINE
\end{lstlisting}

\newpage
\subsubsection{Esercizio 2: Maggiore o Minore di Età}

\begin{tcolorbox}[colback=blue!5!white,colframe=blue!75!black,title=Traccia]
Creare un diagramma di flusso che chieda l'età di una persona e stampi se è maggiorenne (età $\geq$ 18) o minorenne.
\end{tcolorbox}

\textbf{Soluzione:}

\begin{center}
\scalebox{0.8}{
\begin{tikzpicture}[node distance=2cm]
\node (start) [startstop] {Inizio};
\node (input) [io, below of=start] {Leggi eta};
\node (decision) [decision, below of=input, yshift=-0.5cm] {$eta \geq 18$?};
\node (true) [io, left=2.5cm of decision, yshift=-2.5cm] {Scrivi "Maggiorenne"};
\node (false) [io, right=2.5cm of decision, yshift=-2.5cm] {Scrivi "Minorenne"};
\node (stop) [startstop, below=1.5cm of decision, yshift=-2cm] {Fine};

\draw [arrow] (start) -- (input);
\draw [arrow] (input) -- (decision);
\draw [arrow] (decision) -| node[anchor=south east] {Vero} (true);
\draw [arrow] (decision) -| node[anchor=south west] {Falso} (false);
\draw [arrow] (true) |- (stop);
\draw [arrow] (false) |- (stop);
\end{tikzpicture}
}
\end{center}

\textbf{Spiegazione:}
\begin{enumerate}
    \item Il programma chiede di inserire l'età
    \item Verifica se l'età è maggiore o uguale a 18
    \item Se VERO: stampa "Maggiorenne"
    \item Se FALSO: stampa "Minorenne"
    \item Il programma termina
\end{enumerate}

\textbf{Esempi di esecuzione:}
\begin{itemize}
    \item Input: \texttt{eta = 20} → Output: \texttt{Maggiorenne}
    \item Input: \texttt{eta = 16} → Output: \texttt{Minorenne}
    \item Input: \texttt{eta = 18} → Output: \texttt{Maggiorenne}
\end{itemize}

\textbf{Codice equivalente (pseudocodice):}
\begin{lstlisting}
INIZIO
    LEGGI eta
    SE eta >= 18 ALLORA
        SCRIVI "Maggiorenne"
    ALTRIMENTI
        SCRIVI "Minorenne"
    FINE SE
FINE
\end{lstlisting}

\newpage
\subsubsection{Esercizio 3: Massimo tra Due Numeri}

\begin{tcolorbox}[colback=blue!5!white,colframe=blue!75!black,title=Traccia]
Creare un diagramma di flusso che legga due numeri e stampi il maggiore. Se sono uguali, stampare un messaggio appropriato.
\end{tcolorbox}

\textbf{Soluzione:}

\begin{center}
\scalebox{0.8}{
\begin{tikzpicture}[node distance=1.8cm]
\node (start) [startstop] {Inizio};
\node (input1) [io, below of=start] {Leggi a};
\node (input2) [io, below of=input1] {Leggi b};
\node (decision1) [decision, below of=input2, yshift=-0.3cm] {$a > b$?};
\node (out1) [io, left=2.5cm of decision1, yshift=-2cm] {Scrivi "Max: ", a};
\node (decision2) [decision, below=1.5cm of decision1] {$b > a$?};
\node (out2) [io, right=2.5cm of decision2, yshift=-2cm] {Scrivi "Max: ", b};
\node (out3) [io, below=1.5cm of decision2] {Scrivi "Sono uguali"};
\node (stop) [startstop, below=1cm of out3] {Fine};

\draw [arrow] (start) -- (input1);
\draw [arrow] (input1) -- (input2);
\draw [arrow] (input2) -- (decision1);
\draw [arrow] (decision1) -| node[anchor=south east] {Vero} (out1);
\draw [arrow] (decision1) -- node[anchor=east] {Falso} (decision2);
\draw [arrow] (decision2) -| node[anchor=south west] {Vero} (out2);
\draw [arrow] (decision2) -- node[anchor=east] {Falso} (out3);
\draw [arrow] (out1) |- ([yshift=-0.5cm]out1.south) -| (stop);
\draw [arrow] (out2) |- ([yshift=-0.5cm]out2.south) -| (stop);
\draw [arrow] (out3) -- (stop);
\end{tikzpicture}
}
\end{center}

\textbf{Spiegazione:}
\begin{enumerate}
    \item Il programma legge due numeri \texttt{a} e \texttt{b}
    \item Prima verifica se $a > b$:
    \begin{itemize}
        \item Se VERO: \texttt{a} è il maggiore, lo stampa e termina
        \item Se FALSO: procede con la seconda verifica
    \end{itemize}
    \item Seconda verifica se $b > a$:
    \begin{itemize}
        \item Se VERO: \texttt{b} è il maggiore, lo stampa e termina
        \item Se FALSO: i numeri sono uguali, stampa messaggio
    \end{itemize}
\end{enumerate}

\textbf{Esempi di esecuzione:}
\begin{itemize}
    \item Input: \texttt{a=10, b=5} → Output: \texttt{Max: 10}
    \item Input: \texttt{a=3, b=8} → Output: \texttt{Max: 8}
    \item Input: \texttt{a=7, b=7} → Output: \texttt{Sono uguali}
\end{itemize}

\textbf{Codice equivalente (pseudocodice):}
\begin{lstlisting}
INIZIO
    LEGGI a
    LEGGI b
    SE a > b ALLORA
        SCRIVI "Max: ", a
    ALTRIMENTI SE b > a ALLORA
        SCRIVI "Max: ", b
    ALTRIMENTI
        SCRIVI "Sono uguali"
    FINE SE
FINE
\end{lstlisting}

\newpage
\subsubsection{Esercizio 4: Numero Pari o Dispari}

\begin{tcolorbox}[colback=blue!5!white,colframe=blue!75!black,title=Traccia]
Creare un diagramma di flusso che determini se un numero è pari o dispari. Un numero è pari se il resto della divisione per 2 è zero.
\end{tcolorbox}

\textbf{Soluzione:}

\begin{center}
\scalebox{0.8}{
\begin{tikzpicture}[node distance=2cm]
\node (start) [startstop] {Inizio};
\node (input) [io, below of=start] {Leggi numero};
\node (calc) [process, below of=input] {$resto \leftarrow numero \bmod 2$};
\node (decision) [decision, below of=calc, yshift=-0.5cm] {$resto = 0$?};
\node (pari) [io, left=2.5cm of decision, yshift=-2.5cm] {Scrivi "Pari"};
\node (dispari) [io, right=2.5cm of decision, yshift=-2.5cm] {Scrivi "Dispari"};
\node (stop) [startstop, below=1.5cm of decision, yshift=-2cm] {Fine};

\draw [arrow] (start) -- (input);
\draw [arrow] (input) -- (calc);
\draw [arrow] (calc) -- (decision);
\draw [arrow] (decision) -| node[anchor=south east] {Vero} (pari);
\draw [arrow] (decision) -| node[anchor=south west] {Falso} (dispari);
\draw [arrow] (pari) |- (stop);
\draw [arrow] (dispari) |- (stop);
\end{tikzpicture}
}
\end{center}

\textbf{Spiegazione:}
\begin{enumerate}
    \item Il programma legge un numero
    \item Calcola il resto della divisione per 2 usando l'operatore modulo ($\bmod$)
    \item Verifica se il resto è uguale a 0:
    \begin{itemize}
        \item Se VERO: il numero è pari, stampa "Pari"
        \item Se FALSO: il numero è dispari, stampa "Dispari"
    \end{itemize}
\end{enumerate}

\textbf{Concetto matematico:} Un numero è pari se è divisibile per 2 senza resto. L'operatore modulo ($\bmod$) restituisce il resto della divisione.

\textbf{Esempi di esecuzione:}
\begin{itemize}
    \item Input: \texttt{8} → $8 \bmod 2 = 0$ → Output: \texttt{Pari}
    \item Input: \texttt{15} → $15 \bmod 2 = 1$ → Output: \texttt{Dispari}
    \item Input: \texttt{0} → $0 \bmod 2 = 0$ → Output: \texttt{Pari}
\end{itemize}

\textbf{Codice equivalente (pseudocodice):}
\begin{lstlisting}
INIZIO
    LEGGI numero
    resto <- numero MOD 2
    SE resto = 0 ALLORA
        SCRIVI "Pari"
    ALTRIMENTI
        SCRIVI "Dispari"
    FINE SE
FINE
\end{lstlisting}

\newpage
\subsubsection{Esercizio 5: Voto Scolastico con Giudizio}

\begin{tcolorbox}[colback=blue!5!white,colframe=blue!75!black,title=Traccia]
Creare un diagramma di flusso che legga un voto (0-10) e stampi il giudizio corrispondente:
\begin{itemize}
    \item 9-10: Ottimo
    \item 7-8: Buono
    \item 6: Sufficiente
    \item 0-5: Insufficiente
\end{itemize}
\end{tcolorbox}

\textbf{Soluzione:}

\begin{center}
\begin{tikzpicture}[node distance=1.6cm]
\node (start) [startstop] {Inizio};
\node (input) [io, below of=start] {Leggi voto};
\node (d1) [decision, below of=input, yshift=-0.3cm] {$voto \geq 9$?};
\node (o1) [io, right=2cm of d1, yshift=-1.5cm] {Scrivi "Ottimo"};
\node (d2) [decision, below=1.3cm of d1] {$voto \geq 7$?};
\node (o2) [io, right=2cm of d2, yshift=-1.5cm] {Scrivi "Buono"};
\node (d3) [decision, below=1.3cm of d2] {$voto = 6$?};
\node (o3) [io, right=2cm of d3, yshift=-1.5cm] {Scrivi "Sufficiente"};
\node (o4) [io, below=1.3cm of d3] {Scrivi "Insufficiente"};
\node (stop) [startstop, below right=0.8cm and 10cm of o4] {Fine};

\draw [arrow] (start) -- (input);
\draw [arrow] (input) -- (d1);
\draw [arrow] (d1) -| node[anchor=south east, xshift=-0.2cm] {V} (o1);
\draw [arrow] (d1) -- node[anchor=east] {F} (d2);
\draw [arrow] (d2) -| node[anchor=south east, xshift=-0.2cm] {V} (o2);
\draw [arrow] (d2) -- node[anchor=east] {F} (d3);
\draw [arrow] (d3) -| node[anchor=south east, xshift=-0.2cm] {V} (o3);
\draw [arrow] (d3) -- node[anchor=east] {F} (o4);
\draw [arrow] (o1) |- ([yshift=-0.3cm]o1.south) -| (stop);
\draw [arrow] (o2) |- ([yshift=-0.3cm]o2.south) -| (stop);
\draw [arrow] (o3) |- ([yshift=-0.3cm]o3.south) -| (stop);
\draw [arrow] (o4) -- (stop);
\end{tikzpicture}
\end{center}

\textbf{Spiegazione:}
\begin{enumerate}
    \item Il programma legge il voto
    \item Verifica in sequenza le condizioni dall'alto verso il basso:
    \begin{itemize}
        \item Se $voto \geq 9$: stampa "Ottimo"
        \item Altrimenti, se $voto \geq 7$: stampa "Buono"
        \item Altrimenti, se $voto = 6$: stampa "Sufficiente"
        \item Altrimenti: stampa "Insufficiente"
    \end{itemize}
\end{enumerate}

\textbf{Esempi di esecuzione:}
\begin{itemize}
    \item Input: \texttt{10} → Output: \texttt{Ottimo}
    \item Input: \texttt{8} → Output: \texttt{Buono}
    \item Input: \texttt{6} → Output: \texttt{Sufficiente}
    \item Input: \texttt{4} → Output: \texttt{Insufficiente}
\end{itemize}

\textbf{Codice equivalente (pseudocodice):}
\begin{lstlisting}
INIZIO
    LEGGI voto
    SE voto >= 9 ALLORA
        SCRIVI "Ottimo"
    ALTRIMENTI SE voto >= 7 ALLORA
        SCRIVI "Buono"
    ALTRIMENTI SE voto = 6 ALLORA
        SCRIVI "Sufficiente"
    ALTRIMENTI
        SCRIVI "Insufficiente"
    FINE SE
FINE
\end{lstlisting}

\newpage
\subsubsection{Esercizio 6: Calcolatrice Semplice}

\begin{tcolorbox}[colback=blue!5!white,colframe=blue!75!black,title=Traccia]
Creare un diagramma di flusso per una calcolatrice che legga due numeri e un'operazione (+, -, *, /) ed esegua il calcolo corrispondente.
\end{tcolorbox}

\textbf{Soluzione:}

\begin{center}
\begin{tikzpicture}[node distance=1.5cm]
\node (start) [startstop] {Inizio};
\node (in1) [io, below of=start] {Leggi a};
\node (in2) [io, below of=in1] {Leggi b};
\node (in3) [io, below of=in2] {Leggi op};
\node (d1) [decision, below of=in3, yshift=-0.2cm] {$op = '+'$?};
\node (p1) [process, right=1.8cm of d1, yshift=-1.2cm] {$ris \leftarrow a + b$};
\node (d2) [decision, below=1cm of d1] {$op = '-'$?};
\node (p2) [process, right=1.8cm of d2, yshift=-1.2cm] {$ris \leftarrow a - b$};
\node (d3) [decision, below=1cm of d2] {$op = '*'$?};
\node (p3) [process, right=1.8cm of d3, yshift=-1.2cm] {$ris \leftarrow a \times b$};
\node (p4) [process, below=1cm of d3] {$ris \leftarrow a / b$};
\node (out) [io, below right=1.8cm and 6cm of p4] {Scrivi ris};
\node (stop) [startstop, below=1.8cm of out] {Fine};

\draw [arrow] (start) -- (in1);
\draw [arrow] (in1) -- (in2);
\draw [arrow] (in2) -- (in3);
\draw [arrow] (in3) -- (d1);
\draw [arrow] (d1) -| node[anchor=south east, xshift=-0.1cm] {V} (p1);
\draw [arrow] (d1) -- node[anchor=east] {F} (d2);
\draw [arrow] (d2) -| node[anchor=south east, xshift=-0.1cm] {V} (p2);
\draw [arrow] (d2) -- node[anchor=east] {F} (d3);
\draw [arrow] (d3) -| node[anchor=south east, xshift=-0.1cm] {V} (p3);
\draw [arrow] (d3) -- node[anchor=east] {F} (p4);
\draw [arrow] (p1) |- ([yshift=-0.2cm]p1.south) -| (out);
\draw [arrow] (p2) |- ([yshift=-0.2cm]p2.south) -| (out);
\draw [arrow] (p3) |- ([yshift=-0.2cm]p3.south) -| (out);
\draw [arrow] (p4) -- (out);
\draw [arrow] (out) -- (stop);
\end{tikzpicture}
\end{center}

\textbf{Spiegazione:}
\begin{enumerate}
    \item Il programma legge due numeri \texttt{a} e \texttt{b}
    \item Legge l'operazione desiderata \texttt{op}
    \item Verifica quale operazione è stata richiesta:
    \begin{itemize}
        \item Se '+': calcola $ris = a + b$
        \item Se '-': calcola $ris = a - b$
        \item Se '*': calcola $ris = a \times b$
        \item Altrimenti (assume '/'): calcola $ris = a / b$
    \end{itemize}
    \item Stampa il risultato
\end{enumerate}

\textbf{Esempio di esecuzione:}
\begin{itemize}
    \item Input: \texttt{a=10, b=5, op='*'}
    \item Calcolo: $ris = 10 \times 5 = 50$
    \item Output: \texttt{50}
\end{itemize}

\textbf{Nota:} In un programma reale, bisognerebbe verificare che il divisore non sia zero prima di eseguire la divisione.

\textbf{Codice equivalente (pseudocodice):}
\begin{lstlisting}
INIZIO
    LEGGI a, b, op
    SE op = '+' ALLORA
        ris <- a + b
    ALTRIMENTI SE op = '-' ALLORA
        ris <- a - b
    ALTRIMENTI SE op = '*' ALLORA
        ris <- a * b
    ALTRIMENTI
        ris <- a / b
    FINE SE
    SCRIVI ris
FINE
\end{lstlisting}

\newpage
\subsubsection{Esercizio 7: Anno Bisestile}

\begin{tcolorbox}[colback=blue!5!white,colframe=blue!75!black,title=Traccia]
Creare un diagramma di flusso che determini se un anno è bisestile. Un anno è bisestile se:
\begin{itemize}
    \item È divisibile per 4 E non è divisibile per 100
    \item OPPURE è divisibile per 400
\end{itemize}
\end{tcolorbox}

\textbf{Soluzione:}

\begin{center}
\scalebox{0.8}{ % <-- riduce all'80%
\begin{tikzpicture}[node distance=1.8cm]
\node (start) [startstop] {Inizio};
\node (input) [io, below of=start] {Leggi anno};
\node (d1) [decision, below of=input, yshift=-0.3cm, text width=2.5cm] {$anno \bmod 400 = 0$?};
\node (yes1) [io, right=0.5cm of d1, yshift=-1.5cm] {Scrivi "Bisestile"};
\node (d2) [decision,  below left=1.3cm and 1.5cm of d1, text width=2.5cm] {$anno \bmod 100 = 0$?};
\node (no2) [io, right=0.5cm of d2, yshift=-1.5cm] {Scrivi "Non bisestile"};
\node (d3) [decision, below left=1.3cm and 1.5cm of d2, text width=2.5cm] {$anno \bmod 4 = 0$?};
\node (yes3) [io, right=2.5cm of d3, yshift=-1.5cm] {Scrivi "Bisestile"};
\node (no3) [io, below=1.3cm of d3] {Scrivi "Non bisestile"};
\node (stop) [startstop, below right=0.8cm and 15cm of no3] {Fine};

\draw [arrow] (start) -- (input);
\draw [arrow] (input) -- (d1);
\draw [arrow] (d1) -| node[anchor=south east, xshift=-0.1cm] {V} (yes1);
\draw [arrow] (d1) -- node[anchor=east] {F} (d2);
\draw [arrow] (d2) -| node[anchor=south west, xshift=0.1cm] {V} (no2);
\draw [arrow] (d2) -- node[anchor=east] {F} (d3);
\draw [arrow] (d3) -| node[anchor=south east, xshift=-0.1cm] {V} (yes3);
\draw [arrow] (d3) -- node[anchor=east] {F} (no3);
\draw [arrow] (yes1) |- ([yshift=-0.2cm]yes1.south) -| (stop);
\draw [arrow] (no2) |- ([yshift=-0.2cm]no2.south) -| (stop);
\draw [arrow] (yes3) |- ([yshift=-0.2cm]yes3.south) -| (stop);
\draw [arrow] (no3) -- (stop);
\end{tikzpicture}
}
\end{center}

\textbf{Spiegazione della regola:}
\begin{enumerate}
    \item Se l'anno è divisibile per 400 → è sempre bisestile (es. 2000)
    \item Altrimenti, se è divisibile per 100 → NON è bisestile (es. 1900)
    \item Altrimenti, se è divisibile per 4 → è bisestile (es. 2024)
    \item Altrimenti → NON è bisestile (es. 2023)
\end{enumerate}

\textbf{Esempi di esecuzione:}
\begin{itemize}
    \item Input: \texttt{2024} → $2024 \bmod 4 = 0$ (e non per 100) → \texttt{Bisestile}
    \item Input: \texttt{2000} → $2000 \bmod 400 = 0$ → \texttt{Bisestile}
    \item Input: \texttt{1900} → $1900 \bmod 100 = 0$ (ma non per 400) → \texttt{Non bisestile}
    \item Input: \texttt{2023} → $2023 \bmod 4 \neq 0$ → \texttt{Non bisestile}
\end{itemize}

\textbf{Codice equivalente (pseudocodice):}
\begin{lstlisting}
INIZIO
    LEGGI anno
    SE anno MOD 400 = 0 ALLORA
        SCRIVI "Bisestile"
    ALTRIMENTI SE anno MOD 100 = 0 ALLORA
        SCRIVI "Non bisestile"
    ALTRIMENTI SE anno MOD 4 = 0 ALLORA
        SCRIVI "Bisestile"
    ALTRIMENTI
        SCRIVI "Non bisestile"
    FINE SE
FINE
\end{lstlisting}

\newpage
\subsubsection{Esercizio 8: Triangolo Valido e Classificazione}

\begin{tcolorbox}[colback=blue!5!white,colframe=blue!75!black,title=Traccia]
Creare un diagramma di flusso che, dati tre lati, verifichi se formano un triangolo valido e lo classifichi come equilatero, isoscele o scaleno. Un triangolo è valido se la somma di due lati qualsiasi è maggiore del terzo.
\end{tcolorbox}

\textbf{Soluzione:}

\begin{center}
\scalebox{0.7}{
\begin{tikzpicture}[node distance=1.5cm]
\node (start) [startstop] {Inizio};
\node (in1) [io, below of=start] {Leggi a, b, c};
\node (d1) [decision, below of=in1, yshift=-1cm, text width=3cm] {$a+b>c$ AND $a+c>b$ AND $b+c>a$?};
\node (invalid) [io, right=2.5cm of d1, yshift=-1.5cm] {Scrivi "Non valido"};
\node (d2) [decision, below=1.3cm of d1, text width=3cm] {$a=b$ AND $b=c$?};
\node (equi) [io, left=2.5cm of d2, yshift=-1.5cm] {Scrivi "Equilatero"};
\node (d3) [decision, below=1.3cm of d2, text width=3cm] {$a=b$ OR $b=c$ OR $a=c$?};
\node (iso) [io, left=2.5cm of d3, yshift=-1.5cm] {Scrivi "Isoscele"};
\node (sca) [io, below=1.3cm of d3] {Scrivi "Scaleno"};
\node (stop) [startstop, below=0.8cm of sca] {Fine};

\draw [arrow] (start) -- (in1);
\draw [arrow] (in1) -- (d1);
\draw [arrow] (d1) -| node[anchor=south west, xshift=0.1cm] {F} (invalid);
\draw [arrow] (d1) -- node[anchor=east] {V} (d2);
\draw [arrow] (d2) -| node[anchor=south east, xshift=-0.1cm] {V} (equi);
\draw [arrow] (d2) -- node[anchor=east] {F} (d3);
\draw [arrow] (d3) -| node[anchor=south east, xshift=-0.1cm] {V} (iso);
\draw [arrow] (d3) -- node[anchor=east] {F} (sca);
\draw [arrow] (invalid) |- ([yshift=-10.5cm]invalid.south) -| (stop);
\draw [arrow] (equi) -| ([xshift=-3.5cm, yshift=-4.5cm]equi.south) |- (stop);
\draw [arrow] (iso) -| ([xshift=-3.5cm, yshift=-2.75cm]iso.south) -| (stop);
\draw [arrow] (sca) -- (stop);
\end{tikzpicture}
}
\end{center}

\textbf{Spiegazione:}
\begin{enumerate}
    \item Il programma legge i tre lati \texttt{a}, \texttt{b}, \texttt{c}
    \item Verifica se formano un triangolo valido (disuguaglianza triangolare):
    \begin{itemize}
        \item Se NO: stampa "Non valido" e termina
        \item Se SÌ: procede con la classificazione
    \end{itemize}
    \item Verifica se tutti i lati sono uguali → "Equilatero"
    \item Altrimenti verifica se almeno due lati sono uguali → "Isoscele"
    \item Altrimenti → "Scaleno" (tutti i lati diversi)
\end{enumerate}

\textbf{Esempi di esecuzione:}
\begin{itemize}
    \item Input: \texttt{a=5, b=5, c=5} → \texttt{Equilatero}
    \item Input: \texttt{a=5, b=5, c=7} → \texttt{Isoscele}
    \item Input: \texttt{a=3, b=4, c=5} → \texttt{Scaleno}
    \item Input: \texttt{a=1, b=2, c=10} → \texttt{Non valido} ($1+2 \not> 10$)
\end{itemize}

\textbf{Codice equivalente (pseudocodice):}
\begin{lstlisting}
INIZIO
    LEGGI a, b, c
    SE (a+b>c) AND (a+c>b) AND (b+c>a) ALLORA
        SE a=b AND b=c ALLORA
            SCRIVI "Equilatero"
        ALTRIMENTI SE a=b OR b=c OR a=c ALLORA
            SCRIVI "Isoscele"
        ALTRIMENTI
            SCRIVI "Scaleno"
        FINE SE
    ALTRIMENTI
        SCRIVI "Non valido"
    FINE SE
FINE
\end{lstlisting}

\newpage
\subsection{Esercizi Proposti sui Blocchi Condizionali}

Per consolidare le conoscenze sui blocchi condizionali, prova a risolvere i seguenti esercizi:

\begin{enumerate}[leftmargin=*]
    \item \textbf{Sconto sul prezzo}: Creare un diagramma che applichi uno sconto del 10\% se il prezzo è maggiore di 100€, altrimenti nessuno sconto
    
    \item \textbf{Massimo tra tre numeri}: Creare un diagramma che determini il maggiore tra tre numeri
    
    \item \textbf{Ordinamento di due numeri}: Creare un diagramma che ordini due numeri in ordine crescente
    
    \item \textbf{Divisibilità}: Creare un diagramma che verifichi se un numero è divisibile per 3 e per 5 contemporaneamente
    
    \item \textbf{Equazione di secondo grado}: Creare un diagramma che, dati i coefficienti $a$, $b$, $c$, determini se l'equazione $ax^2 + bx + c = 0$ ha soluzioni reali, controllando il discriminante $\Delta = b^2 - 4ac$
    
    \item \textbf{Categoria peso}: Creare un diagramma che, dato il peso di una persona, la classifichi come sottopeso ($<50$ kg), normopeso ($50-80$ kg) o sovrappeso ($>80$ kg)
    
    \item \textbf{Segno del prodotto}: Creare un diagramma che, dati due numeri, determini se il loro prodotto è positivo, negativo o zero senza calcolare il prodotto
    
    \item \textbf{Conversione voto}: Creare un diagramma che converta un voto in centesimi (0-100) in trentesimi (0-30), con lode se il voto è maggiore o uguale a 99
    
    \item \textbf{Accesso a servizio}: Creare un diagramma che verifichi se un utente può accedere a un servizio: deve avere età $\geq 18$ E (saldo $\geq 10$ O abbonamento attivo)
    
    \item \textbf{Stagione}: Creare un diagramma che, dato il numero del mese (1-12), stampi la stagione corrispondente
\end{enumerate}

\vspace{1cm}
\begin{tcolorbox}[colback=green!10!white,colframe=green!75!black,title=Suggerimento]
Quando risolvi esercizi con condizioni multiple, ricorda:
\begin{itemize}
    \item AND: Entrambe le condizioni devono essere vere
    \item OR: Almeno una delle condizioni deve essere vera
    \item Usa le parentesi per chiarire la priorità delle operazioni
    \item Disegna prima i casi limite (valori estremi) per verificare la logica
\end{itemize}
\end{tcolorbox}

\newpage
\section{Cicli (Iterazione)}

\subsection{Cosa sono i Cicli}

I \textbf{cicli} (o strutture iterative) permettono di ripetere un blocco di istruzioni più volte, fino a quando una determinata condizione è soddisfatta. Sono fondamentali per automatizzare operazioni ripetitive senza dover scrivere lo stesso codice molte volte.

\subsubsection{Perché usare i cicli?}

I cicli sono essenziali quando dobbiamo:
\begin{itemize}[leftmargin=*]
    \item Eseguire la stessa operazione su molti dati
    \item Contare o sommare valori
    \item Cercare un elemento in una sequenza
    \item Validare input fino a quando è corretto
    \item Elaborare dati fino a una condizione di stop
\end{itemize}

\subsection{Tipi di Cicli}

Esistono tre tipi principali di cicli, ognuno adatto a situazioni diverse:

\subsubsection{Ciclo WHILE (pre-condizionale)}

Il ciclo \textbf{while} verifica la condizione \textit{prima} di eseguire il blocco di istruzioni. Se la condizione è falsa fin dall'inizio, il blocco non viene mai eseguito.

\textbf{Sintassi:}
\begin{lstlisting}
MENTRE condizione vera FARE
    istruzioni
    TORNA A MENTRE
\end{lstlisting}

\textbf{Caratteristiche:}
\begin{itemize}[leftmargin=*]
    \item Controllo \textit{prima} dell'esecuzione
    \item Può eseguire 0 o più iterazioni
    \item Usato quando non si sa a priori quante volte ripetere
\end{itemize}

\begin{center}
\begin{tikzpicture}[node distance=2cm]
\node (start) [startstop] {Inizio};
\node (init) [process, below of=start] {Inizializzazione};
\node (decision) [decision, below of=init, yshift=-0.5cm] {Condizione?};
\node (process) [process, left=2.5cm of decision, yshift=-1cm] {Istruzioni};
\node (stop) [startstop, right=2cm of decision] {Fine};

\draw [arrow] (start) -- (init);
\draw [arrow] (init) -- (decision);
\draw [arrow] (decision) |- ([yshift=-2.75cm]decision.south) -| (process) 
      node[midway, above] {Vero};
\draw [arrow] (process) |- (decision);
\draw [arrow] (decision) -- node[anchor=east] {Falso} (stop);
\end{tikzpicture}
\end{center}

\newpage
\subsubsection{Ciclo DO-WHILE (post-condizionale)}

Il ciclo \textbf{do-while} esegue il blocco di istruzioni \textit{prima} di verificare la condizione. Garantisce almeno un'esecuzione del blocco.

\textbf{Sintassi:}
\begin{lstlisting}
FARE
    istruzioni
MENTRE condizione
\end{lstlisting}

\textbf{Caratteristiche:}
\begin{itemize}[leftmargin=*]
    \item Controllo \textit{dopo} l'esecuzione
    \item Esegue almeno 1 volta (anche se la condizione è falsa)
    \item Usato quando si deve eseguire almeno una volta (es. menu, validazione input)
\end{itemize}

\begin{center}
\begin{tikzpicture}[node distance=2cm]
\node (start) [startstop] {Inizio};
\node (init) [process, below of=start] {Inizializzazione};
\node (process) [process, below of=init] {Istruzioni};
\node (decision) [decision, below of=process, yshift=-0.5cm] {Condizione?};
\node (stop) [startstop, below=2cm of decision] {Fine};

\draw [arrow] (start) -- (init);
\draw [arrow] (init) -- (process);
\draw [arrow] (process) -- (decision);
\draw [arrow] (decision) -| node[anchor=south, near start] {Vero} ++(-3,0) |- (process);
\draw [arrow] (decision) -- node[anchor=east] {Falso} (stop);
\end{tikzpicture}
\end{center}

\subsubsection{Ciclo FOR (a contatore)}

Il ciclo \textbf{for} è usato quando si conosce esattamente il numero di iterazioni da eseguire. Utilizza una variabile contatore.

\textbf{Sintassi:}
\begin{lstlisting}
PER i DA valore_iniziale A valore_finale [PASSO incremento] FARE
    istruzioni
FINE PER
\end{lstlisting}

\textbf{Caratteristiche:}
\begin{itemize}[leftmargin=*]
    \item Numero di iterazioni noto a priori
    \item Gestisce automaticamente il contatore
    \item Usato per iterare su intervalli numerici
\end{itemize}

\begin{center}
\begin{tikzpicture}[node distance=2cm]
\node (start) [startstop] {Inizio};
\node (init) [process, below of=start, text width=3.5cm] {$i \leftarrow valore\_iniziale$};
\node (decision) [decision, below of=init, yshift=-0.5cm, text width=2.8cm] {$i \leq valore\_finale$?};
\node (process) [process, below right=2.5cm and 1cm of decision] {Istruzioni};
\node (increment) [process, below of=process] {$i \leftarrow i + passo$};
\node (stop) [startstop, below=2cm of decision] {Fine};

\draw [arrow] (start) -- (init);
\draw [arrow] (init) -- (decision);
\draw [arrow] (decision) -- node[anchor=south] {Vero} (process);
\draw [arrow] (process) -- (increment);
\draw [arrow] (increment) -| ++(2cm,0) |- (decision);
\draw [arrow] (decision) -- node[anchor=east] {Falso} (stop);
\end{tikzpicture}
\end{center}

\newpage
\subsection{Componenti Fondamentali di un Ciclo}

Ogni ciclo ben strutturato deve avere:

\begin{enumerate}[leftmargin=*]
    \item \textbf{Inizializzazione}: Preparare le variabili prima del ciclo
    \item \textbf{Condizione di controllo}: Determina se continuare o terminare
    \item \textbf{Corpo del ciclo}: Le istruzioni da ripetere
    \item \textbf{Aggiornamento}: Modificare le variabili per avvicinarsi alla fine
\end{enumerate}

\begin{tcolorbox}[colback=red!10!white,colframe=red!75!black,title=Attenzione: Cicli Infiniti!]
Un \textbf{ciclo infinito} si verifica quando la condizione di uscita non diventa mai falsa. Questo causa il blocco del programma!

\textbf{Esempio di errore:}
\begin{lstlisting}
i <- 0
MENTRE i < 10 FARE
    SCRIVI i
    // ERRORE: manca i <- i + 1
FINE MENTRE
\end{lstlisting}

Assicurati sempre che il ciclo modifichi le variabili della condizione!
\end{tcolorbox}

\subsection{Confronto tra i Cicli}

\begin{center}
\begin{tabular}{|p{3cm}|p{4cm}|p{4cm}|p{3cm}|}
\hline
\textbf{Tipo} & \textbf{Controllo} & \textbf{Quando usarlo} & \textbf{Iterazioni} \\
\hline
WHILE & Prima & Numero iterazioni sconosciuto & 0 o più \\
\hline
DO-WHILE & Dopo & Almeno una esecuzione necessaria & 1 o più \\
\hline
FOR & Prima & Numero iterazioni noto & 0 o più \\
\hline
\end{tabular}
\end{center}

\newpage
\subsection{Esercizi Svolti con Cicli}

\subsubsection{Esercizio 1: Contare da 1 a 10 (Ciclo FOR)}

\begin{tcolorbox}[colback=blue!5!white,colframe=blue!75!black,title=Traccia]
Creare un diagramma di flusso che stampi i numeri da 1 a 10 utilizzando un ciclo FOR.
\end{tcolorbox}

\textbf{Soluzione:}

\begin{center}
\begin{tikzpicture}[node distance=2cm]
\node (start) [startstop] {Inizio};
\node (init) [process, below of=start] {$i \leftarrow 1$};
\node (decision) [decision, below of=init, yshift=-0.5cm] {$i \leq 10$?};
\node (output) [io, below right=1cm and 2cm of decision] {Scrivi i};
\node (increment) [process, below of=output] {$i \leftarrow i + 1$};
\node (stop) [startstop, below=2cm of decision] {Fine};

\draw [arrow] (start) -- (init);
\draw [arrow] (init) -- (decision);
\draw [arrow] (decision) -- node[anchor=south] {V} (output);
\draw [arrow] (output) -- (increment);
\draw [arrow] (increment) -| ++(2cm,0) |- (decision);
\draw [arrow] (decision) -- node[anchor=east] {F} (stop);
\end{tikzpicture}
\end{center}

\textbf{Spiegazione:}
\begin{enumerate}
    \item Inizializza il contatore $i$ a 1
    \item Verifica se $i \leq 10$:
    \begin{itemize}
        \item Se VERO: stampa il valore di $i$, incrementa $i$ di 1 e torna alla verifica
        \item Se FALSO: esce dal ciclo e termina
    \end{itemize}
\end{enumerate}

\textbf{Traccia di esecuzione:}
\begin{itemize}
    \item Iterazione 1: $i=1$, stampa 1, $i$ diventa 2
    \item Iterazione 2: $i=2$, stampa 2, $i$ diventa 3
    \item ... (continua fino a $i=10$)
    \item Iterazione 10: $i=10$, stampa 10, $i$ diventa 11
    \item Verifica: $11 \leq 10$ è FALSO → esce dal ciclo
\end{itemize}

\textbf{Output:} \texttt{1 2 3 4 5 6 7 8 9 10}

\textbf{Codice equivalente (pseudocodice):}
\begin{lstlisting}
INIZIO
    PER i DA 1 A 10 FARE
        SCRIVI i
    FINE PER
FINE
\end{lstlisting}

\newpage
\subsubsection{Esercizio 2: Somma dei primi N numeri (Ciclo WHILE)}

\begin{tcolorbox}[colback=blue!5!white,colframe=blue!75!black,title=Traccia]
Creare un diagramma di flusso che calcoli la somma dei primi N numeri naturali (da 1 a N), dove N è inserito dall'utente.
\end{tcolorbox}

\textbf{Soluzione:}

\begin{center}
\begin{tikzpicture}[node distance=2cm]
\node (start) [startstop] {Inizio};
\node (input) [io, below of=start] {Leggi N};
\node (init1) [process, below of=input] {$somma \leftarrow 0$};
\node (init2) [process, below of=init1] {$i \leftarrow 1$};
\node (decision) [decision, below of=init2, yshift=-0.5cm] {$i \leq N$?};
\node (calc) [process, below right=1cm and 2.5cm of decision, yshift=-0.5cm] {$somma \leftarrow somma + i$};
\node (increment) [process, below of=calc] {$i \leftarrow i + 1$};
\node (output) [io, below=2.5cm of decision] {Scrivi somma};
\node (stop) [startstop, below of=output] {Fine};

\draw [arrow] (start) -- (input);
\draw [arrow] (input) -- (init1);
\draw [arrow] (init1) -- (init2);
\draw [arrow] (init2) -- (decision);
\draw [arrow] (decision) -- node[anchor=south] {V} (calc);
\draw [arrow] (calc) -- (increment);
\draw [arrow] (increment) -| ++(3cm,0) |-  (decision);
\draw [arrow] (decision) -- node[anchor=east] {F} (output);
\draw [arrow] (output) -- (stop);
\end{tikzpicture}
\end{center}

\textbf{Spiegazione:}
\begin{enumerate}
    \item L'utente inserisce il valore di N
    \item Inizializza $somma = 0$ e $i = 1$
    \item Finché $i \leq N$:
    \begin{itemize}
        \item Aggiunge $i$ alla somma
        \item Incrementa $i$ di 1
    \end{itemize}
    \item Stampa il risultato finale
\end{enumerate}

\textbf{Esempio di esecuzione (N=5):}
\begin{itemize}
    \item Input: $N = 5$
    \item Inizializzazione: $somma = 0$, $i = 1$
    \item Iter. 1: $somma = 0 + 1 = 1$, $i = 2$
    \item Iter. 2: $somma = 1 + 2 = 3$, $i = 3$
    \item Iter. 3: $somma = 3 + 3 = 6$, $i = 4$
    \item Iter. 4: $somma = 6 + 4 = 10$, $i = 5$
    \item Iter. 5: $somma = 10 + 5 = 15$, $i = 6$
    \item Verifica: $6 \leq 5$ è FALSO → Output: \texttt{15}
\end{itemize}

\textbf{Formula matematica:} $\sum_{i=1}^{N} i = \frac{N(N+1)}{2}$

\textbf{Codice equivalente (pseudocodice):}
\begin{lstlisting}
INIZIO
    LEGGI N
    somma <- 0
    i <- 1
    MENTRE i <= N FARE
        somma <- somma + i
        i <- i + 1
    FINE MENTRE
    SCRIVI somma
FINE
\end{lstlisting}

\newpage
\subsubsection{Esercizio 3: Fattoriale di un Numero (Ciclo FOR)}

\begin{tcolorbox}[colback=blue!5!white,colframe=blue!75!black,title=Traccia]
Creare un diagramma di flusso che calcoli il fattoriale di un numero N (N! = 1 × 2 × 3 × ... × N).
\end{tcolorbox}

\textbf{Soluzione:}

\begin{center}
\begin{tikzpicture}[node distance=2cm]
\node (start) [startstop] {Inizio};
\node (input) [io, below of=start] {Leggi N};
\node (init1) [process, below of=input] {$fattoriale \leftarrow 1$};
\node (init2) [process, below of=init1] {$i \leftarrow 1$};
\node (decision) [decision, below of=init2, yshift=-0.5cm] {$i \leq N$?};
\node (calc) [process, right=2.8cm of decision, yshift=-0.5cm] {$fattoriale \leftarrow fattoriale \times i$};
\node (increment) [process, below of=calc] {$i \leftarrow i + 1$};
\node (output) [io, below=2.5cm of decision] {Scrivi fattoriale};
\node (stop) [startstop, below of=output] {Fine};

\draw [arrow] (start) -- (input);
\draw [arrow] (input) -- (init1);
\draw [arrow] (init1) -- (init2);
\draw [arrow] (init2) -- (decision);
\draw [arrow] (decision) -- node[anchor=south] {V} (calc);
\draw [arrow] (calc) -- (increment);
\draw [arrow] (increment) |- (decision);
\draw [arrow] (decision) -- node[anchor=east] {F} (output);
\draw [arrow] (output) -- (stop);
\end{tikzpicture}
\end{center}

\textbf{Spiegazione:}
\begin{enumerate}
    \item L'utente inserisce il numero N
    \item Inizializza $fattoriale = 1$ (elemento neutro della moltiplicazione)
    \item Per ogni $i$ da 1 a N:
    \begin{itemize}
        \item Moltiplica $fattoriale$ per $i$
    \end{itemize}
    \item Stampa il risultato
\end{enumerate}

\textbf{Esempio di esecuzione (N=5):}
\begin{itemize}
    \item Input: $N = 5$
    \item Inizializzazione: $fattoriale = 1$
    \item Iter. 1: $fattoriale = 1 \times 1 = 1$
    \item Iter. 2: $fattoriale = 1 \times 2 = 2$
    \item Iter. 3: $fattoriale = 2 \times 3 = 6$
    \item Iter. 4: $fattoriale = 6 \times 4 = 24$
    \item Iter. 5: $fattoriale = 24 \times 5 = 120$
    \item Output: \texttt{120} (5! = 120)
\end{itemize}

\textbf{Nota:} Il fattoriale cresce molto rapidamente: 10! = 3,628,800

\textbf{Codice equivalente (pseudocodice):}
\begin{lstlisting}
INIZIO
    LEGGI N
    fattoriale <- 1
    PER i DA 1 A N FARE
        fattoriale <- fattoriale * i
    FINE PER
    SCRIVI fattoriale
FINE
\end{lstlisting}

\newpage
\subsubsection{Esercizio 4: Validazione Input (Ciclo DO-WHILE)}

\begin{tcolorbox}[colback=blue!5!white,colframe=blue!75!black,title=Traccia]
Creare un diagramma di flusso che chieda all'utente di inserire un numero compreso tra 1 e 10. Se il numero non è valido, continua a chiederlo fino a quando non viene inserito un valore corretto.
\end{tcolorbox}

\textbf{Soluzione:}

\begin{center}
\begin{tikzpicture}[node distance=2cm]
\node (start) [startstop] {Inizio};
\node (input) [io, below of=start] {Leggi numero};
\node (decision) [decision, below of=input, yshift=-0.5cm, text width=3cm] {$numero < 1$ OR $numero > 10$?};
\node (error) [io, right=2.5cm of decision, yshift=-1.5cm] {Scrivi "Errore! Riprova"};
\node (success) [io, below=2cm of decision] {Scrivi "Valido!"};
\node (stop) [startstop, below of=success] {Fine};

\draw [arrow] (start) -- (input);
\draw [arrow] (input) -- (decision);
\draw [arrow] (decision) -- node[anchor=south] {V} (error);
\draw [arrow] (error) |- ++(0,3) -| (input);
\draw [arrow] (decision) -- node[anchor=east] {F} (success);
\draw [arrow] (success) -- (stop);
\end{tikzpicture}
\end{center}

\textbf{Spiegazione:}
\begin{enumerate}
    \item Il programma chiede di inserire un numero
    \item Verifica se il numero è fuori dall'intervallo [1, 10]:
    \begin{itemize}
        \item Se VERO (numero non valido): stampa messaggio di errore e torna all'input
        \item Se FALSO (numero valido): stampa "Valido!" e termina
    \end{itemize}
\end{enumerate}

\textbf{Perché DO-WHILE?} Almeno un tentativo di input deve essere fatto, quindi il ciclo DO-WHILE è perfetto per la validazione.

\textbf{Esempio di esecuzione:}
\begin{itemize}
    \item Tentativo 1: utente inserisce 15 → "Errore! Riprova"
    \item Tentativo 2: utente inserisce -3 → "Errore! Riprova"
    \item Tentativo 3: utente inserisce 7 → "Valido!" → fine
\end{itemize}

\textbf{Codice equivalente (pseudocodice):}
\begin{lstlisting}
INIZIO
    FARE
        LEGGI numero
        SE numero < 1 OR numero > 10 ALLORA
            SCRIVI "Errore! Riprova"
        FINE SE
    MENTRE numero < 1 OR numero > 10
    SCRIVI "Valido!"
FINE
\end{lstlisting}

\newpage
\subsubsection{Esercizio 5: Tavola Pitagorica (Cicli Annidati)}

\begin{tcolorbox}[colback=blue!5!white,colframe=blue!75!black,title=Traccia]
Creare un diagramma di flusso che stampi la tavola pitagorica del 5 (5×1, 5×2, ... 5×10) usando un ciclo FOR.
\end{tcolorbox}

\textbf{Soluzione:}

\begin{center}
\begin{tikzpicture}[node distance=2cm]
\node (start) [startstop] {Inizio};
\node (const) [process, below of=start] {$numero \leftarrow 5$};
\node (init) [process, below of=const] {$i \leftarrow 1$};
\node (decision) [decision, below of=init, yshift=-0.5cm] {$i \leq 10$?};
\node (calc) [process, right=2.5cm of decision, yshift=-0.5cm] {$risultato \leftarrow numero \times i$};
\node (output) [io, below of=calc, text width=3.5cm] {Scrivi numero, "x", i, "=", risultato};
\node (increment) [process, below of=output] {$i \leftarrow i + 1$};
\node (stop) [startstop, below=3cm of decision] {Fine};

\draw [arrow] (start) -- (const);
\draw [arrow] (const) -- (init);
\draw [arrow] (init) -- (decision);
\draw [arrow] (decision) -- node[anchor=south] {V} (calc);
\draw [arrow] (calc) -- (output);
\draw [arrow] (output) -- (increment);
\draw [arrow] (increment) |- (decision);
\draw [arrow] (decision) -- node[anchor=east] {F} (stop);
\end{tikzpicture}
\end{center}

\textbf{Spiegazione:}
\begin{enumerate}
    \item Imposta il numero base (5)
    \item Per ogni $i$ da 1 a 10:
    \begin{itemize}
        \item Calcola $risultato = numero \times i$
        \item Stampa l'operazione e il risultato
    \end{itemize}
\end{enumerate}

\textbf{Output:}
\begin{verbatim}
5 x 1 = 5
5 x 2 = 10
5 x 3 = 15
5 x 4 = 20
5 x 5 = 25
5 x 6 = 30
5 x 7 = 35
5 x 8 = 40
5 x 9 = 45
5 x 10 = 50
\end{verbatim}

\textbf{Codice equivalente (pseudocodice):}
\begin{lstlisting}
INIZIO
    numero <- 5
    PER i DA 1 A 10 FARE
        risultato <- numero * i
        SCRIVI numero, " x ", i, " = ", risultato
    FINE PER
FINE
\end{lstlisting}

\newpage
\subsubsection{Esercizio 6: Contare i Numeri Pari (Ciclo WHILE)}

\begin{tcolorbox}[colback=blue!5!white,colframe=blue!75!black,title=Traccia]
Creare un diagramma di flusso che conti quanti numeri pari ci sono tra 1 e N (dove N è inserito dall'utente).
\end{tcolorbox}

\textbf{Soluzione:}

\begin{center}
\begin{tikzpicture}[node distance=1.8cm]
\node (start) [startstop] {Inizio};
\node (input) [io, below of=start] {Leggi N};
\node (init1) [process, below of=input] {$contatore \leftarrow 0$};
\node (init2) [process, below of=init1] {$i \leftarrow 1$};
\node (d1) [decision, below of=init2, yshift=-0.3cm] {$i \leq N$?};
\node (d2) [decision, right=2.3cm of d1, yshift=-1cm] {$i \bmod 2 = 0$?};
\node (count) [process, right=1.5cm of d2] {$contatore \leftarrow contatore + 1$};
\node (incr) [process, below=1.5cm of d2] {$i \leftarrow i + 1$};
\node (output) [io, below=2.5cm of d1] {Scrivi contatore};
\node (stop) [startstop, below of=output] {Fine};

\draw [arrow] (start) -- (input);
\draw [arrow] (input) -- (init1);
\draw [arrow] (init1) -- (init2);
\draw [arrow] (init2) -- (d1);
\draw [arrow] (d1) -- node[anchor=south] {V} (d2);
\draw [arrow] (d2) -- node[anchor=south] {V} (count);
\draw [arrow] (d2) -- node[anchor=east] {F} (incr);
\draw [arrow] (count) |- (incr);
\draw [arrow] (incr) -- ++(0,-0.5) -| (d1);
\draw [arrow] (d1) -- node[anchor=east] {F} (output);
\draw [arrow] (output) -- (stop);
\end{tikzpicture}
\end{center}

\textbf{Spiegazione:}
\begin{enumerate}
    \item L'utente inserisce N
    \item Inizializza il contatore a 0 e $i$ a 1
    \item Per ogni numero da 1 a N:
    \begin{itemize}
        \item Se il numero è pari (resto della divisione per 2 è zero), incrementa il contatore
        \item Passa al numero successivo
    \end{itemize}
    \item Stampa quanti numeri pari sono stati trovati
\end{enumerate}

\textbf{Esempio di esecuzione (N=10):}
\begin{itemize}
    \item Input: $N = 10$
    \item Numeri pari trovati: 2, 4, 6, 8, 10
    \item Output: \texttt{5} (ci sono 5 numeri pari tra 1 e 10)
\end{itemize}

\textbf{Formula diretta:} Per N pari: $\frac{N}{2}$, per N dispari: $\frac{N-1}{2}$

\textbf{Codice equivalente (pseudocodice):}
\begin{lstlisting}
INIZIO
    LEGGI N
    contatore <- 0
    i <- 1
    MENTRE i <= N FARE
        SE i MOD 2 = 0 ALLORA
            contatore <- contatore + 1
        FINE SE
        i <- i + 1
    FINE MENTRE
    SCRIVI contatore
FINE
\end{lstlisting}

\newpage
\subsubsection{Esercizio 7: Media di N Numeri (Ciclo WHILE)}

\begin{tcolorbox}[colback=blue!5!white,colframe=blue!75!black,title=Traccia]
Creare un diagramma di flusso che calcoli la media di N numeri inseriti dall'utente. Prima chiede quanti numeri verranno inseriti, poi li legge uno alla volta e infine calcola la media.
\end{tcolorbox}

\textbf{Soluzione:}

\begin{center}
\begin{tikzpicture}[node distance=1.8cm]
\node (start) [startstop] {Inizio};
\node (in1) [io, below of=start] {Leggi N};
\node (init1) [process, below of=in1] {$somma \leftarrow 0$};
\node (init2) [process, below of=init1] {$i \leftarrow 1$};
\node (d1) [decision, below of=init2, yshift=-0.3cm] {$i \leq N$?};
\node (in2) [io, right=2.5cm of d1] {Leggi numero};
\node (calc) [process, below of=in2] {$somma \leftarrow somma + numero$};
\node (incr) [process, below of=calc] {$i \leftarrow i + 1$};
\node (media) [process, below=2.5cm of d1] {$media \leftarrow somma / N$};
\node (output) [io, below of=media] {Scrivi media};
\node (stop) [startstop, below of=output] {Fine};

\draw [arrow] (start) -- (in1);
\draw [arrow] (in1) -- (init1);
\draw [arrow] (init1) -- (init2);
\draw [arrow] (init2) -- (d1);
\draw [arrow] (d1) -- node[anchor=south] {V} (in2);
\draw [arrow] (in2) -- (calc);
\draw [arrow] (calc) -- (incr);
\draw [arrow] (incr) |- (d1);
\draw [arrow] (d1) -- node[anchor=east] {F} (media);
\draw [arrow] (media) -- (output);
\draw [arrow] (output) -- (stop);
\end{tikzpicture}
\end{center}

\textbf{Spiegazione:}
\begin{enumerate}
    \item Chiede quanti numeri verranno inseriti (N)
    \item Inizializza la somma a 0
    \item Per N volte:
    \begin{itemize}
        \item Legge un numero
        \item Lo aggiunge alla somma
    \end{itemize}
    \item Calcola la media: $media = somma / N$
    \item Stampa la media
\end{enumerate}

\textbf{Esempio di esecuzione:}
\begin{itemize}
    \item Input: $N = 4$
    \item Input numeri: 8, 6, 7, 9
    \item Somma: $8 + 6 + 7 + 9 = 30$
    \item Media: $30 / 4 = 7.5$
    \item Output: \texttt{7.5}
\end{itemize}

\textbf{Codice equivalente (pseudocodice):}
\begin{lstlisting}
INIZIO
    LEGGI N
    somma <- 0
    i <- 1
    MENTRE i <= N FARE
        LEGGI numero
        somma <- somma + numero
        i <- i + 1
    FINE MENTRE
    media <- somma / N
    SCRIVI media
FINE
\end{lstlisting}

\newpage
\subsubsection{Esercizio 8: Sequenza di Fibonacci (Ciclo FOR)}

\begin{tcolorbox}[colback=blue!5!white,colframe=blue!75!black,title=Traccia]
Creare un diagramma di flusso che stampi i primi N numeri della sequenza di Fibonacci, dove ogni numero è la somma dei due precedenti (1, 1, 2, 3, 5, 8, 13, ...).
\end{tcolorbox}

\textbf{Soluzione:}

\begin{center}
\scalebox{0.85}{
\begin{tikzpicture}[node distance=1.8cm]
\node (start) [startstop] {Inizio};
\node (input) [io, below of=start] {Leggi N};
\node (init1) [process, below of=input] {$a \leftarrow 1, b \leftarrow 1$};
\node (out1) [io, below of=init1] {Scrivi a};
\node (check) [decision, below of=out1, yshift=-0.3cm] {$N > 1$?};
\node (out2) [io, below of=check] {Scrivi b};
\node (init2) [process, below of=out2] {$i \leftarrow 3$};
\node (d1) [decision, below of=init2, yshift=-0.3cm] {$i \leq N$?};
\node (calc) [process, right=2.5cm of d1, yshift=-0.5cm] {$temp \leftarrow a + b$};
\node (out3) [io, below of=calc] {Scrivi temp};
\node (update) [process, below of=out3] {$a \leftarrow b, b \leftarrow temp$};
\node (incr) [process, below of=update] {$i \leftarrow i + 1$};
\node (stop) [startstop, below=3.5cm of d1] {Fine};

\draw [arrow] (start) -- (input);
\draw [arrow] (input) -- (init1);
\draw [arrow] (init1) -- (out1);
\draw [arrow] (out1) -- (check);
\draw [arrow] (check) -- node[anchor=east] {V} (out2);
\draw [arrow] (check) -| node[anchor=south, near start] {F} ++(3,0) |- (stop);
\draw [arrow] (out2) -- (init2);
\draw [arrow] (init2) -- (d1);
\draw [arrow] (d1) -- node[anchor=south] {V} (calc);
\draw [arrow] (calc) -- (out3);
\draw [arrow] (out3) -- (update);
\draw [arrow] (update) -- (incr);
\draw [arrow] (incr) |- (d1);
\draw [arrow] (d1) -- node[anchor=east] {F} (stop);
\end{tikzpicture}
}
\end{center}

\textbf{Spiegazione:}
\begin{enumerate}
    \item Chiede quanti numeri di Fibonacci stampare
    \item Inizializza i primi due numeri: $a=1, b=1$
    \item Stampa il primo numero
    \item Se $N > 1$, stampa anche il secondo numero
    \item Per i numeri successivi (da 3 a N):
    \begin{itemize}
        \item Calcola il nuovo numero: $temp = a + b$
        \item Stampa $temp$
        \item Aggiorna: $a = b$ e $b = temp$
    \end{itemize}
\end{enumerate}

\textbf{Esempio di esecuzione (N=8):}
\begin{itemize}
    \item Output: 1, 1, 2, 3, 5, 8, 13, 21
    \item Calcoli: $1+1=2$, $1+2=3$, $2+3=5$, $3+5=8$, $5+8=13$, $8+13=21$
\end{itemize}

\textbf{Codice equivalente (pseudocodice):}
\begin{lstlisting}
INIZIO
    LEGGI N
    a <- 1, b <- 1
    SCRIVI a
    SE N > 1 ALLORA
        SCRIVI b
        PER i DA 3 A N FARE
            temp <- a + b
            SCRIVI temp
            a <- b
            b <- temp
        FINE PER
    FINE SE
FINE
\end{lstlisting}

\newpage
\subsection{Esercizi Proposti sui Cicli}

Per consolidare le conoscenze sui cicli, prova a risolvere i seguenti esercizi:

\begin{enumerate}[leftmargin=*]
    \item \textbf{Numeri dispari}: Creare un diagramma che stampi tutti i numeri dispari da 1 a 50 (Ciclo FOR)
    
    \item \textbf{Conto alla rovescia}: Creare un diagramma che conti alla rovescia da 10 a 0 (Ciclo FOR)
    
    \item \textbf{Somma fino a zero}: Creare un diagramma che continui a leggere numeri e sommarli finché l'utente non inserisce 0, poi stampi la somma totale (Ciclo WHILE)
    
    \item \textbf{Massimo di N numeri}: Creare un diagramma che trovi il numero più grande tra N numeri inseriti dall'utente (Ciclo WHILE)
    
    \item \textbf{Password}: Creare un diagramma che chieda una password e continui a richiederla finché non viene inserita quella corretta "1234" (Ciclo DO-WHILE)
    
    \item \textbf{Potenza}: Creare un diagramma che calcoli $base^{esponente}$ usando un ciclo (es. $2^5 = 2 \times 2 \times 2 \times 2 \times 2$) (Ciclo FOR)
    
    \item \textbf{Divisori}: Creare un diagramma che stampi tutti i divisori di un numero N (Ciclo FOR)
    
    \item \textbf{Numero primo}: Creare un diagramma che verifichi se un numero è primo (divisibile solo per 1 e se stesso) (Ciclo WHILE)
    
    \item \textbf{MCD}: Creare un diagramma che calcoli il Massimo Comun Divisore tra due numeri usando l'algoritmo di Euclide (Ciclo WHILE)
    
    \item \textbf{Quadrato di asterischi}: Creare un diagramma che stampi un quadrato di asterischi di dimensione N×N (Cicli annidati FOR)
\end{enumerate}

\vspace{1cm}
\begin{tcolorbox}[colback=green!10!white,colframe=green!75!black,title=Suggerimenti per i Cicli]
\begin{itemize}
    \item \textbf{Inizializzazione}: Imposta sempre i valori iniziali delle variabili prima del ciclo
    \item \textbf{Condizione di uscita}: Assicurati che la condizione diventi falsa prima o poi
    \item \textbf{Aggiornamento}: Ricorda di modificare le variabili nel corpo del ciclo
    \item \textbf{Test con casi limite}: Prova il ciclo con 0, 1 e molte iterazioni
    \item \textbf{Traccia manuale}: Esegui il ciclo passo-passo su carta per verificare la logica
    \item \textbf{Cicli annidati}: Fai attenzione a usare variabili diverse per cicli esterni e interni
\end{itemize}
\end{tcolorbox}

\newpage
\section{Appendice: Convenzioni e Suggerimenti}

\subsection{Convenzioni di Scrittura}

\begin{itemize}[leftmargin=*]
    \item \textbf{Nomi delle variabili}: Usare nomi significativi (es. \texttt{prezzo}, \texttt{eta}, \texttt{somma})
    \item \textbf{Assegnazione}: Si può usare sia $\leftarrow$ che $=$ (es. $x \leftarrow 5$ oppure $x = 5$)
    \item \textbf{Operatori matematici}: 
    \begin{itemize}
        \item Somma: $+$
        \item Sottrazione: $-$
        \item Moltiplicazione: $\times$ o $*$
        \item Divisione: $/$
        \item Modulo (resto): $\%$ o $mod$
    \end{itemize}
    \item \textbf{Operatori di confronto}: $=, \neq, <, >, \leq, \geq$
\end{itemize}

\subsection{Suggerimenti per Disegnare Diagrammi}

\begin{enumerate}[leftmargin=*]
    \item \textbf{Pianifica prima di disegnare}: Scrivi prima l'algoritmo in pseudocodice
    \item \textbf{Mantieni la semplicità}: Un blocco = un'operazione semplice
    \item \textbf{Usa spaziature uniformi}: Mantieni distanze regolari tra i blocchi
    \item \textbf{Allinea i blocchi}: Cerca di mantenere un allineamento verticale
    \item \textbf{Etichetta le frecce}: Nelle decisioni, indica sempre Vero/Falso o Sì/No
    \item \textbf{Verifica il flusso}: Segui mentalmente il percorso per verificare la correttezza
    \item \textbf{Testa con esempi}: Prova il diagramma con valori concreti
\end{enumerate}

\subsection{Errori Comuni da Evitare}

\begin{itemize}[leftmargin=*]
    \item Dimenticare di inizializzare le variabili prima di usarle
    \item Confondere l'operatore di assegnazione ($\leftarrow$) con quello di confronto ($=$)
    \item Non specificare la direzione Vero/Falso nei rombi decisionali
    \item Creare cicli infiniti senza condizione di uscita
    \item Lasciare percorsi senza sbocco (che non arrivano alla fine)
    \item Usare lo stesso nome per variabili diverse
    \item Fare operazioni con variabili di tipo incompatibile
\end{itemize}

\end{document}
