\documentclass[a4paper,12pt]{article}
\usepackage[utf8]{inputenc}
\usepackage[italian]{babel}
\usepackage{amsmath}
\usepackage{amssymb}
\usepackage{graphicx}
\usepackage{circuitikz}
\usepackage{geometry}
\usepackage{xcolor}
\geometry{margin=2.5cm}

\title{Introduzione all'elettrotecnica}
\author{Appunti di Elettrotecnica}
\date{}

\begin{document}

\maketitle
\tableofcontents

\section{\textcolor{gray}{Fondamenti: Tensione e Corrente}}

\subsection{Corrente Elettrica}

\subsubsection{Definizione}

La \textbf{corrente elettrica} è il flusso ordinato di cariche elettriche attraverso un conduttore. Rappresenta la quantità di carica che attraversa una sezione del conduttore nell'unità di tempo.

\textbf{Definizione matematica:}
\begin{equation}
I = \frac{Q}{t}
\end{equation}

dove:
\begin{itemize}
    \item $I$ = Corrente elettrica (Ampere, A)
    \item $Q$ = Carica elettrica (Coulomb, C)
    \item $t$ = Tempo (secondi, s)
\end{itemize}

\textbf{Unità di misura:}
\begin{itemize}
    \item \textbf{Ampere} (A): unità fondamentale del Sistema Internazionale
    \item $1\,\mathrm{A} = 1\,\mathrm{C/s}$ (un coulomb al secondo)
    \item Sottomultipli comuni:
    \begin{itemize}
        \item milliampere: $1\,\mathrm{mA} = 10^{-3}\,\mathrm{A}$
        \item microampere: $1\,\mu\mathrm{A} = 10^{-6}\,\mathrm{A}$
    \end{itemize}
\end{itemize}

\subsubsection{Natura Fisica}

La corrente elettrica è costituita dal movimento di:
\begin{itemize}
    \item \textbf{Elettroni} nei conduttori metallici (verso opposto alla corrente convenzionale)
    \item \textbf{Ioni} nelle soluzioni elettrolitiche
    \item \textbf{Lacune} ed elettroni nei semiconduttori
\end{itemize}

\textbf{Convenzione:} La corrente convenzionale va dal polo positivo al polo negativo (direzione opposta al movimento degli elettroni).

\subsubsection{Tipi di Corrente}

\textbf{Corrente Continua (DC):}
\begin{itemize}
    \item Flusso costante in intensità e direzione
    \item Esempio: batterie, alimentatori DC
    \item Simbolo: $\equiv$ o DC
\end{itemize}

\textbf{Corrente Alternata (AC):}
\begin{itemize}
    \item Flusso variabile periodicamente nel tempo
    \item Esempio: rete elettrica domestica
    \item Simbolo: $\sim$ o AC
\end{itemize}

\subsection{Tensione Elettrica}

\subsubsection{Definizione}

La \textbf{tensione elettrica} (o differenza di potenziale) è l'energia necessaria per spostare una carica elettrica tra due punti di un circuito. Rappresenta la "spinta" che muove le cariche elettriche.

\textbf{Definizione matematica:}
\begin{equation}
V = \frac{W}{Q}
\end{equation}

dove:
\begin{itemize}
    \item $V$ = Tensione (Volt, V)
    \item $W$ = Lavoro o Energia (Joule, J)
    \item $Q$ = Carica elettrica (Coulomb, C)
\end{itemize}

\textbf{Unità di misura:}
\begin{itemize}
    \item \textbf{Volt} (V): unità derivata del Sistema Internazionale
    \item $1\,\mathrm{V} = 1\,\mathrm{J/C}$ (un joule per coulomb)
    \item Multipli e sottomultipli comuni:
    \begin{itemize}
        \item kilovolt: $1\,\mathrm{kV} = 10^{3}\,\mathrm{V}$
        \item millivolt: $1\,\mathrm{mV} = 10^{-3}\,\mathrm{V}$
        \item microvolt: $1\,\mu\mathrm{V} = 10^{-6}\,\mathrm{V}$
    \end{itemize}
\end{itemize}

\subsubsection{Natura Fisica}

La tensione elettrica:
\begin{itemize}
    \item È sempre una \textbf{differenza di potenziale} tra due punti
    \item Rappresenta l'energia per unità di carica
    \item È analoga alla pressione in un sistema idraulico
    \item Esiste anche in assenza di corrente (circuito aperto)
\end{itemize}

\textbf{Analogia idraulica:}
\begin{itemize}
    \item Tensione $\leftrightarrow$ Pressione dell'acqua
    \item Corrente $\leftrightarrow$ Flusso d'acqua
    \item Resistenza $\leftrightarrow$ Restringimento del tubo
\end{itemize}

\subsection{Relazione tra Tensione e Corrente}

\subsubsection{Potenza Elettrica}

La potenza elettrica è il prodotto tra tensione e corrente:

\begin{equation}
P = V \cdot I
\end{equation}

dove:
\begin{itemize}
    \item $P$ = Potenza (Watt, W)
    \item $V$ = Tensione (Volt, V)
    \item $I$ = Corrente (Ampere, A)
\end{itemize}

\textbf{Interpretazione:}
\begin{itemize}
    \item La potenza rappresenta l'energia trasferita nell'unità di tempo
    \item $1\,\mathrm{W} = 1\,\mathrm{V} \cdot 1\,\mathrm{A}$
    \item Nei resistori, la potenza è sempre dissipata (trasformata in calore)
\end{itemize}

\subsubsection{Esempi Pratici}

\textbf{Esempio 0.1:} Una batteria eroga una carica di $360\,\mathrm{C}$ in $2$ minuti. Calcolare la corrente.

\textit{Soluzione:}
\begin{align*}
t &= 2\,\mathrm{min} = 120\,\mathrm{s} \\
I &= \frac{Q}{t} = \frac{360}{120} = 3\,\mathrm{A}
\end{align*}

\textbf{Esempio 0.2:} Per spostare una carica di $0.5\,\mathrm{C}$ tra due punti sono necessari $6\,\mathrm{J}$ di energia. Calcolare la tensione.

\textit{Soluzione:}
\begin{align*}
V &= \frac{W}{Q} = \frac{6}{0.5} = 12\,\mathrm{V}
\end{align*}

\textbf{Esempio 0.3:} Una lampadina funziona a $230\,\mathrm{V}$ e assorbe una corrente di $0.26\,\mathrm{A}$. Calcolare la potenza dissipata.

\textit{Soluzione:}
\begin{align*}
P &= V \cdot I = 230 \cdot 0.26 = 59.8\,\mathrm{W} \approx 60\,\mathrm{W}
\end{align*}

\subsection{Misura di Tensione e Corrente}

\subsubsection{Strumenti di Misura}

\textbf{Voltmetro:}
\begin{itemize}
    \item Misura la tensione (differenza di potenziale)
    \item Si collega in \textbf{parallelo} al componente
    \item Resistenza interna molto alta (idealmente infinita)
    \item Non deve alterare la corrente del circuito
\end{itemize}

\textbf{Amperometro:}
\begin{itemize}
    \item Misura la corrente
    \item Si collega in \textbf{serie} al circuito
    \item Resistenza interna molto bassa (idealmente nulla)
    \item Non deve alterare la tensione del circuito
\end{itemize}

\subsubsection{Rappresentazione Grafica}

\begin{center}
\begin{circuitikz}[scale=1.3]
    % Circuito con voltmetro e amperometro
    \draw (0,0) to[battery1, v=$12\,\mathrm{V}$] (0,3)
          to[short] (1,3)
          to[ammeter, l=$A$, i>^=$I$] (3,3)
          to[R=$R$, v=$V_R$] (5,3)
          to[short] (5,0)
          to[short] (0,0);
    
    % Voltmetro in parallelo
    \draw (5,3) to[short] (6,3)
          to[voltmeter, l=$V$] (6,0)
          to[short] (5,0);
    
    \node at (3,-0.7) {Amperometro in serie};
    \node at (6.8,1.5) {Voltmetro};
    \node at (6.8,1) {in parallelo};
\end{circuitikz}
\end{center}

\subsection{Riepilogo delle Grandezze Fondamentali}

\begin{center}
\begin{tabular}{|l|c|c|c|}
\hline
\textbf{Grandezza} & \textbf{Simbolo} & \textbf{Unità} & \textbf{Formula} \\
\hline
Carica elettrica & $Q$ & Coulomb (C) & - \\
\hline
Corrente & $I$ & Ampere (A) & $I = \frac{Q}{t}$ \\
\hline
Tensione & $V$ & Volt (V) & $V = \frac{W}{Q}$ \\
\hline
Resistenza & $R$ & Ohm ($\Omega$) & $R = \frac{V}{I}$ \\
\hline
Potenza & $P$ & Watt (W) & $P = V \cdot I$ \\
\hline
\end{tabular}
\end{center}

\newpage

\section{Legge di Ohm}

\subsection{Teoria}
La legge di Ohm è una delle leggi fondamentali dell'elettrotecnica e stabilisce la relazione tra tensione, corrente e resistenza in un circuito elettrico.

\textbf{Formula:}
\begin{equation}
V = R \cdot I
\end{equation}

dove:
\begin{itemize}
    \item $V$ = Tensione (Volt)
    \item $R$ = Resistenza (Ohm, $\Omega$)
    \item $I$ = Corrente (Ampere, A)
\end{itemize}

Dalla formula principale si possono ricavare:
\begin{align}
I &= \frac{V}{R} \\
R &= \frac{V}{I}
\end{align}

\subsection{Esercizi Svolti}

\textbf{Esercizio 1.1:} Calcolare la corrente che attraversa una resistenza di $100\,\Omega$ alimentata da una tensione di $12\,V$.

\textit{Soluzione:}
\begin{align*}
I &= \frac{V}{R} = \frac{12\,V}{100\,\Omega} = 0.12\,A = 120\,mA
\end{align*}

\textbf{Esercizio 1.2:} Una lampadina attraversata da una corrente di $0.5\,A$ ha ai suoi capi una tensione di $230\,V$. Calcolare la resistenza della lampadina.

\textit{Soluzione:}
\begin{align*}
R &= \frac{V}{I} = \frac{230\,V}{0.5\,A} = 460\,\Omega
\end{align*}

\textbf{Esercizio 1.3:} Calcolare la tensione ai capi di un resistore da $2.2\,k\Omega$ attraversato da una corrente di $5\,mA$.

\textit{Soluzione:}
\begin{align*}
V &= R \cdot I = 2200\,\Omega \cdot 0.005\,A = 11\,V
\end{align*}

\newpage
\section{Componenti Fondamentali dei Circuiti}

\subsection{Teoria}

\subsubsection{Generatore di Tensione Ideale}

Un \textbf{generatore di tensione ideale} è un dispositivo che mantiene una differenza di potenziale costante ai suoi capi, indipendentemente dalla corrente che lo attraversa.

\textbf{Caratteristiche:}
\begin{itemize}
    \item Fornisce una tensione costante $V$
    \item La corrente erogata dipende dal carico collegato
    \item Resistenza interna nulla (ideale)
    \item Simbolo: batteria o generatore
\end{itemize}

\textbf{Rappresentazione grafica:}

\begin{center}
\begin{circuitikz}[scale=1.5]
    % Generatore di tensione con batteria
    \draw (0,0) to[battery1, invert, v=$V$, i<^=$I$] (0,2);
    \node at (0,-0.8) {Batteria};
    
    % Generatore di tensione generico
    \draw (3,0) to[V, v=$V$, i<^=$I$] (3,2);
    \node at (3,-0.8) {Generatore};
    
    % Con segni espliciti
    \draw (6,0) to[battery1, invert, v=$V$, i<^=$I$] (6,2);
    \node at (5.5,2) {$+$};
    \node at (5.5,0) {$-$};
    \node at (6,-0.8) {Con polarità};
\end{circuitikz}
\end{center}

\textbf{Equazione caratteristica:}
\begin{equation}
V = \text{costante}
\end{equation}

\textbf{Comportamento:}
\begin{itemize}
    \item A circuito aperto: $I = 0$, $V = V_{nominale}$
    \item Con carico: $V = V_{nominale}$, $I = \frac{V}{R_{carico}}$
    \item In cortocircuito (teorico): $I \to \infty$ (nella realtà limitato dalla resistenza interna)
\end{itemize}

\subsubsection{Generatore di Corrente Ideale}

Un \textbf{generatore di corrente ideale} è un dispositivo che eroga una corrente costante, indipendentemente dalla tensione ai suoi capi.

\textbf{Caratteristiche:}
\begin{itemize}
    \item Fornisce una corrente costante $I$
    \item La tensione ai suoi capi dipende dal carico collegato
    \item Resistenza interna infinita (ideale)
    \item Meno comune nella pratica rispetto al generatore di tensione
\end{itemize}

\textbf{Rappresentazione grafica:}

\begin{center}
\begin{circuitikz}[scale=1.5]
    % Generatore di corrente
    \draw (0,0) to[I, i=$I$, v=$V$] (0,2);
    \node at (0,-0.8) {Simbolo standard};
\end{circuitikz}
\end{center}

\textbf{Equazione caratteristica:}
\begin{equation}
I = \text{costante}
\end{equation}

\textbf{Comportamento:}
\begin{itemize}
    \item A circuito aperto: $V \to \infty$ (teorico), $I = I_{nominale}$
    \item Con carico: $I = I_{nominale}$, $V = R_{carico} \cdot I$
    \item In cortocircuito: $V = 0$, $I = I_{nominale}$
\end{itemize}

\subsubsection{Resistore}

Un \textbf{resistore} è un componente passivo che si oppone al passaggio della corrente elettrica, dissipando energia sotto forma di calore.

\textbf{Caratteristiche:}
\begin{itemize}
    \item Valore espresso in Ohm ($\Omega$)
    \item Componente passivo (non genera energia)
    \item Segue la Legge di Ohm: $V = R \cdot I$
    \item Dissipa potenza: $P = V \cdot I = R \cdot I^2 = \frac{V^2}{R}$
\end{itemize}

\textbf{Rappresentazione grafica:}

\begin{center}
\begin{circuitikz}[scale=1.5]
    % Resistore standard
    \draw (0,0) to[R=$R$, v=$V$, i=$I$] (3,0);
    \node at (1.5,-0.8) {Simbolo standard};
    
    % Resistore con valore
    \draw (4,0) to[R=$100\,\Omega$, v=$V$, i=$I$] (7,0);
    \node at (5.5,-0.8) {Con valore};
\end{circuitikz}
\end{center}

\textbf{Equazione caratteristica (Legge di Ohm):}
\begin{equation}
V = R \cdot I
\end{equation}

\textbf{Comportamento:}
\begin{itemize}
    \item Relazione lineare tra tensione e corrente
    \item La potenza dissipata è sempre positiva
    \item Non dipende dal verso della corrente (componente simmetrico)
\end{itemize}

\subsubsection{Confronto tra i Componenti}

\begin{center}
\begin{tabular}{|l|c|c|c|}
\hline
\textbf{Caratteristica} & \textbf{Gen. Tensione} & \textbf{Gen. Corrente} & \textbf{Resistore} \\
\hline
Tipo & Attivo & Attivo & Passivo \\
\hline
Grandezza costante & $V$ & $I$ & $R$ \\
\hline
Relazione $V$-$I$ & $V = $ cost. & $I = $ cost. & $V = R \cdot I$ \\
\hline
Resistenza interna & $0\,\Omega$ (ideale) & $\infty\,\Omega$ (ideale) & $R$ \\
\hline
Potenza & Fornita & Fornita & Dissipata \\
\hline
\end{tabular}
\end{center}

\subsubsection{Esempi di Circuiti con i Tre Componenti}

\textbf{Esempio 1: Circuito con Generatore di Tensione}

\begin{center}
\begin{circuitikz}[scale=1.3]
    \draw (0,0) to[battery1, invert, v=$12\,V$, i=$I$] (0,3)
          to[short] (1,3)
          to[R=$100\,\Omega$, v<=$V_R$, i>^=$I$] (4,3)
          to[short] (4,0)
          to[short] (0,0);

\end{circuitikz}
\end{center}

Calcoli:
\begin{align*}
I &= \frac{V}{R} = \frac{12}{100} = 0.12\,A = 120\,mA \\
V_R &= 12\,V \quad \text{(tutta la tensione cade sulla resistenza)} \\
P_R &= V \cdot I = 12 \cdot 0.12 = 1.44\,W \quad \text{(dissipata)}
\end{align*}

\textbf{Esempio 2: Circuito con Generatore di Corrente}

\begin{center}
\begin{circuitikz}[scale=1.3]
    \draw (0,0) to[I, l=$50\,mA$, i=$I$, v=$V_g$] (0,3)
          to[short] (1,3)
          to[R=$200\,\Omega$, v<=$V_R$, i>^=$I$] (4,3)
          to[short] (4,0)
          to[short] (0,0);
\end{circuitikz}
\end{center}

Calcoli:
\begin{align*}
I &= 50\,mA = 0.05\,A \quad \text{(costante)} \\
V_R &= R \cdot I = 200 \cdot 0.05 = 10\,V \\
V_g &= V_R = 10\,V \quad \text{(tensione ai capi del generatore)} \\
P_R &= V_R \cdot I = 10 \cdot 0.05 = 0.5\,W
\end{align*}

\textbf{Esempio 3: Confronto tra Generatori}

\begin{center}
\begin{circuitikz}[scale=1.2]
    % Generatore di tensione
    \draw (0,0) to[battery1, v=$10\,V$] (0,2)
          to[short] (0.5,2)
          to[R=$R$, i=$I_V$] (2.5,2)
          to[short] (3,2)
          to[short] (3,0)
          to[short] (0,0);
    \node at (1.5,-0.5) {Gen. Tensione};
    \node at (1.5,1) {$I_V = \frac{10}{R}$};
    
    % Generatore di corrente
    \draw (5,0) to[I, l=$1\,A$] (5,2)
          to[short] (5.5,2)
          to[R=$R$, i=$I_I$] (7.5,2)
          to[short] (8,2)
          to[short] (8,0)
          to[short] (5,0);
    \node at (6.5,-0.5) {Gen. Corrente};
    \node at (6.5,1) {$V = R \cdot 1$};
\end{circuitikz}
\end{center}

\textbf{Osservazioni:}
\begin{itemize}
    \item Con il generatore di tensione, la corrente dipende da $R$
    \item Con il generatore di corrente, la tensione dipende da $R$
    \item Il resistore si comporta identicamente in entrambi i casi
\end{itemize}

\newpage
\section{Nodi nei Circuiti Elettrici}

\subsection{Teoria}

\subsubsection{Definizione di Nodo}
Un \textbf{nodo} è un punto di connessione in un circuito elettrico dove si incontrano due o più componenti. Più precisamente, un nodo è un punto (o insieme di punti collegati da conduttori ideali) dove convergono almeno tre rami del circuito. Ad ogni nodo corrisponde un potenziale, cioè una certa quantita di energia "ipotetica" msiurata in V.

\subsubsection{Come Riconoscere i Nodi}

Per identificare correttamente i nodi in un circuito:

\begin{enumerate}
    \item \textbf{Punti di giunzione}: Cercare i punti dove si collegano tre o più elementi
    \item \textbf{Conduttori ideali}: Tutti i punti collegati da un filo (senza resistenze intermedie) formano lo stesso nodo
    \item \textbf{Non sono propriamente nodi}: I punti dove si collegano solo due elementi non sono propriamente nodi (semplice passaggio di corrente), ma si possono trattare anche quelli come nodi
\end{enumerate}

\textbf{Proprietà dei nodi:}
\begin{itemize}
    \item In un nodo, la somma algebrica delle correnti è zero (Prima Legge di Kirchhoff o KCL)
    \item Tutti i punti di un nodo hanno lo stesso potenziale elettrico
    \item I nodi sono fondamentali per l'analisi dei circuiti
\end{itemize}

\subsubsection{Esempi di Identificazione dei Nodi}

\begin{center}
\begin{circuitikz}[scale=1.2]
    \draw (0,0) to[battery1, v=$V$] (0,3)
          to[short] (1,3)
          to[R=$R_1$] (3,3) node[circ, label=above:\textbf{Nodo A}] {}
          to[R=$R_2$] (3,1.5)
          to[short] (3,0)
          to[short] (0,0);
    \draw (3,3) to[short] (5,3)
          to[R=$R_3$] (5,1.5)
          to[short] (5,0)
          to[short] (3,0);
    \draw (3,3) to[short] (7,3)
          to[R=$R_4$] (7,1.5)
          to[short] (7,0) node[circ, label=below:\textbf{Nodo B}] {}
          to[short] (5,0);
\end{circuitikz}
\end{center}

In questo circuito misto:
\begin{itemize}
    \item \textbf{Nodo A}: Punto superiore dove si dividono le correnti verso $R_2$, $R_3$ e $R_4$ (4 rami: da $R_1$ e verso le tre resistenze in parallelo)
    \item \textbf{Nodo B}: Punto inferiore dove si ricongiungono le correnti (4 rami: dalle tre resistenze in parallelo e verso la batteria)
    \item Totale: 2 nodi principali
\end{itemize}

\subsubsection{Definizione di Maglia}

Una \textbf{maglia} (o ciclo) è un percorso chiuso in un circuito elettrico che:
\begin{itemize}
    \item Parte da un nodo e ritorna allo stesso nodo
    \item Non passa due volte per lo stesso ramo
    \item Forma un percorso continuo attraverso i componenti del circuito
\end{itemize}

\textbf{Come Riconoscere le Maglie:}

\begin{enumerate}
    \item Identificare un punto di partenza (un nodo qualsiasi)
    \item Seguire un percorso attraverso i componenti del circuito
    \item Verificare di tornare al punto di partenza senza ripercorrere lo stesso ramo
    \item Ogni percorso chiuso distinto costituisce una maglia diversa
\end{enumerate}

\textbf{Esempi di Identificazione delle Maglie}

\textbf{Esempio 1: Circuito Serie (1 maglia)}

\begin{center}
\begin{circuitikz}[scale=1.2]
    \draw[very thick, blue] (0,0) to[battery1, v<=$V$] (0,3)
          to[short] (1,3)
          to[R=$R_1$] (3,3)
          to[R=$R_2$] (5,3)
          to[R=$R_3$] (7,3)
          to[short] (7,0)
          to[short] (0,0);
    \node[blue] at (3.5,1.5) {\Large Maglia 1};
\end{circuitikz}
\end{center}

In questo circuito c'è una sola maglia che include la batteria e tutte e tre le resistenze.

\textbf{Esempio 2: Circuito Misto (2 maglie principali)}

\begin{center}
\begin{circuitikz}[scale=1.2]
    \draw (0,0) to[battery1, v=$V$] (0,3)
          to[short] (1,3)
          to[R=$R_1$] (3,3);
    
    % Maglia sinistra (blu)
    \draw[very thick, blue] (3,3) to[short] (3,2.5)
          to[R=$R_2$] (3,1)
          to[short] (3,0)
          to[short] (0,0);
    
    % Maglia destra (rosso)
    \draw[very thick, red] (3,3) to[short] (5,3)
          to[R=$R_3$] (5,1)
          to[short] (5,0)
          to[short] (3,0);
    
    \node[blue] at (1.5,1.5) {Maglia A};
    \node[red] at (4,1.5) {Maglia B};
\end{circuitikz}
\end{center}

In questo circuito misto:
\begin{itemize}
    \item \textbf{Maglia A} (blu): Batteria $\rightarrow$ $R_1$ $\rightarrow$ $R_2$ $\rightarrow$ Batteria
    \item \textbf{Maglia B} (rossa): Batteria $\rightarrow$ $R_1$ $\rightarrow$ $R_3$ $\rightarrow$ Batteria
    \item Esiste anche una maglia interna $R_2$-$R_3$, ma è combinazione delle prime due
\end{itemize}

\textbf{Proprietà importanti delle maglie:}
\begin{itemize}
    \item Il numero di maglie indipendenti in un circuito è dato da: $M = R - N + 1$ dove $R$ è il numero di rami e $N$ il numero di nodi
    \item Le maglie sono fondamentali per applicare la Seconda Legge di Kirchhoff (KVL)
    \item In circuiti complessi, scegliere le maglie giuste semplifica l'analisi
\end{itemize}

\subsubsection{Prima Legge di Kirchhoff (KCL)}

Nei nodi vale la \textbf{Legge di Kirchhoff delle Correnti}:

\begin{equation}
\sum_{k=1}^{n} I_k = 0
\end{equation}

ovvero: \textit{La somma algebrica delle correnti entranti in un nodo è uguale alla somma delle correnti uscenti}.

\textbf{Esempio applicativo sul circuito parallelo precedente (Nodo A):}
\begin{align*}
I_{entrante} &= I_1 + I_2 + I_3 \\
\text{oppure:}\quad I_{tot} &= I_1 + I_2 + I_3
\end{align*}

dove $I_{tot}$ entra nel nodo e $I_1$, $I_2$, $I_3$ escono verso le rispettive resistenze.

\subsubsection{Seconda Legge di Kirchhoff (KVL)}

Oltre alla legge delle correnti, Kirchhoff formulò anche la \textbf{Legge delle Tensioni} (KVL - Kirchhoff's Voltage Law):

\begin{equation}
\sum_{k=1}^{n} V_k = 0
\end{equation}

ovvero: \textit{La somma algebrica delle tensioni lungo una maglia chiusa è uguale a zero}.

\textbf{Definizione di maglia:}
Una maglia è un percorso chiuso in un circuito che parte da un punto e ritorna allo stesso punto senza passare due volte per lo stesso ramo.

\textbf{Come applicare la KVL:}
\begin{enumerate}
    \item Scegliere un verso di percorrenza della maglia (orario o antiorario)
    \item Assegnare il segno positivo alle tensioni che si incontrano dal $+$ al $-$ seguendo il verso scelto
    \item Assegnare il segno negativo alle tensioni che si incontrano dal $-$ al $+$
    \item La somma algebrica deve essere zero
\end{enumerate}

\textbf{Esempio applicativo sul circuito serie:}

In un circuito serie con una batteria $V$ e tre resistenze $R_1$, $R_2$, $R_3$:

\begin{equation}
V - V_1 - V_2 - V_3 = 0
\end{equation}

oppure:

\begin{equation}
V = V_1 + V_2 + V_3
\end{equation}

dove la tensione della batteria è positiva (genera tensione) e le cadute di tensione sulle resistenze sono negative (consumano tensione).

\textbf{Proprietà della KVL:}
\begin{itemize}
    \item Vale per qualsiasi percorso chiuso in un circuito
    \item È indipendente dal verso di percorrenza scelto
    \item È fondamentale per l'analisi delle maglie nei circuiti complessi
    \item Deriva dal principio di conservazione dell'energia
\end{itemize}

\subsection{Assegnazione di Tensioni e Correnti nei Circuiti}

\subsubsection{Convenzioni Fondamentali}

Quando si analizza un circuito elettrico, è fondamentale assegnare correttamente tensioni e correnti seguendo convenzioni standard.

\textbf{Convenzione per le Correnti:}

\begin{itemize}
    \item La corrente si assegna con un \textbf{verso arbitrario} prima di risolvere il circuito
    \item Si indica con una freccia sul ramo del circuito
    \item Se il risultato del calcolo è \textbf{positivo}, il verso assegnato è corretto
    \item Se il risultato è \textbf{negativo}, il verso reale è opposto a quello assegnato
    \item Nei \textbf{generatori}, la corrente esce dal polo positivo (interno al generatore)
    \item Nei \textbf{resistori}, la corrente fluisce dal potenziale maggiore al minore
\end{itemize}

\textbf{Convenzione per le Tensioni:}

\begin{itemize}
    \item La tensione si indica con i segni $+$ e $-$ ai capi del componente
    \item Per i \textbf{generatori}: il polo $+$ è quello a potenziale maggiore
    \item Per i \textbf{resistori}: il $+$ si mette dove entra la corrente (convenzione degli utilizzatori)
    \item La tensione si misura sempre \textbf{tra due punti} (differenza di potenziale)
\end{itemize}

\subsubsection{Convenzione degli Utilizzatori}

Per i componenti passivi (resistori), si usa la \textbf{convenzione degli utilizzatori}:

\begin{center}
\begin{circuitikz}[scale=1.5]
    % Resistenza con corrente verso sinistra
    \draw (0,0) to[R, v=$V_R$, i<=$I$, l=$R$] (3,0);
    
    % Segni della tensione (positivo a destra, negativo a sinistra)
    \node at (-0.3,0) {$-$};
    \node at (3.3,0) {$+$};
    
    % Freccia della corrente verso sinistra
    \draw[<-, very thick, blue] (1.5,0.5) -- (2.5,0.5) node[midway, above] {verso $I$};
\end{circuitikz}
\end{center}

\textbf{Regola:} La corrente entra dal polo positivo ed esce dal polo negativo.

\subsubsection{Convenzione dei generatori}

Per i generatori (batterie, generatori di tensione), si usa la \textbf{convenzione dei generatori}: la corrente convenzionale esce dal polo positivo del generatore e rientra nel polo negativo.

\begin{center}
\begin{circuitikz}[scale=1.5]
    % Generatore con corrente uscente dal positivo
    \draw (0,2) to[battery1, v<=$V_g$, i<=$I$] (0,0);
    
    % Segni del generatore
    \node at (-0.5,2) {$+$};
    \node at (-0.5,0) {$-$};
    
    % Freccia della corrente uscente dal positivo
    \draw[->, very thick, red] (0.3,1) -- (0.3,1.8) node[midway, right] {verso $I$};
\end{circuitikz}
\end{center}


\textbf{Regola:} Internamente al generatore, la corrente va dal polo $-$ al polo $+$ (il generatore "pompa" cariche).

Nel circuito esterno, la corrente esce dal $-$ ed entra nel $+$.

\subsubsection{Esempi Pratici}

\textbf{Esempio 1: Circuito Serie - Assegnazione Completa}

\begin{center}
\begin{circuitikz}[scale=1.3]
    % Batteria con segni e corrente
    \draw (0,0) to[battery1, invert, v=$V_g{=}12\,V$, i>=$I$] (0,3);
    \node at (-0.8,3) {$+$};
    \node at (-0.8,0) {$-$};
    
    % R1
    \draw (0,3) to[short] (1,3)
          to[R=$R_1$, v=$V_1$, i>^=$I$] (3,3);
    \node at (0.8,3.3) {$-$};
    \node at (2.8,3.3) {$+$};
    
    % R2
    \draw (3,3) to[R=$R_2$, v=$V_2$, i>^=$I$] (5,3);
    \node at (3.2,3.3) {$-$};
    \node at (5.2,3.3) {$+$};
    
    % Chiusura
    \draw (5,3) to[short] (5,0)
          to[short] (0,0);
    
    % Frecce versi
    \draw[->, very thick, blue] (0.5,3.4) -- (1.5,3.4) node[midway, above] {\small verso corrente};
\end{circuitikz}
\end{center}

\textbf{Spiegazione:}
\begin{itemize}
    \item La corrente $I$ esce dal polo $+$ della batteria
    \item Attraversa $R_1$ e $R_2$ nello stesso verso
    \item Su $R_1$: il $+$ è a sinistra (dove entra $I$), il $-$ a destra
    \item Su $R_2$: il $+$ è a sinistra (dove entra $I$), il $-$ a destra
    \item Vale la KVL: $V_g - V_1 - V_2 = 0 \Rightarrow V_g = V_1 + V_2$
\end{itemize}

\textbf{Calcoli:}
Se $R_1 = 100\,\Omega$ e $R_2 = 200\,\Omega$:
\begin{align*}
R_{eq} &= 300\,\Omega \\
I &= \frac{12}{300} = 0.04\,A = 40\,mA
V_1 &= 100 \cdot 0.04 = 4\,V \\
V_2 &= 200 \cdot 0.04 = 8\,V
\end{align*}

\textbf{Esempio 2: Circuito Parallelo - Divisione Correnti}

\begin{center}
\begin{circuitikz}[scale=1.3]
    % Batteria
    \draw (-0.8,3) to[battery1, v<=$V{=}9\,V$, i<=$I_{tot}$] (-0.8,0);
    \node at (-1.8,3) {$+$};
    \node at (-1.8,0) {$-$};
    
    % Nodo A superiore
    \draw (-0.8,3) to[short, i>=$I_{tot}$] (2,3);
    \node[circ, label=above:A] at (2,3) {};
    
    % R1
    \draw (2,3) to[short, i=$I_1$] (2,2.7)
          to[R=$R_1$, v<=$V$, i>^=$I_1$] (2,0.3)
          to[short] (2,0);
    \node at (1.5,2.5) {$+$};
    \node at (1.5,0.5) {$-$};
    
    % R2
    \draw (2,3) to[short, i=$I_2$] (4,3)
          to[R=$R_2$, v<=$V$, i>^=$I_2$] (4,0)
          to[short] (2,0);
    \node at (3.5,2.5) {$+$};
    \node at (3.5,0.5) {$-$};
    
    % R3
    \draw (4,3) to[short, i=$I_3$] (6,3)
          to[R=$R_3$, v<=$V$, i>^=$I_3$] (6,0)
          to[short] (4,0);
    \node at (5.5,2.5) {$+$};
    \node at (5.5,0.5) {$-$};
    
    % Nodo B inferiore
    \node[circ, label=below:B] at (2,0) {};
    \draw (2,0) to[short, i>=$I_{tot}$] (-0.8,0);
\end{circuitikz}
\end{center}

\textbf{Spiegazione:}
\begin{itemize}
    \item Al nodo A: $I_{tot} = I_1 + I_2 + I_3$ (KCL)
    \item Tutte le resistenze hanno la stessa tensione $V = 9\,V$
    \item Su ogni resistore: il $+$ è in alto (nodo A), il $-$ in basso (nodo B)
    \item Ogni corrente fluisce dall'alto (potenziale maggiore) verso il basso
\end{itemize}

\textbf{Calcoli:}
Se $R_1 = 90\,\Omega$, $R_2 = 180\,\Omega$, $R_3 = 270\,\Omega$:
\begin{align*}
I_1 &= \frac{9}{90} = 100\,mA \quad \text{(verso corretto: verso il basso)} \\
I_2 &= \frac{9}{180} = 50\,mA \\
I_3 &= \frac{9}{270} = 33.3\,mA \\
I_{tot} &= 100 + 50 + 33.3 = 183.3\,mA
\end{align*}

\subsubsection{Riepilogo delle Regole}

\begin{enumerate}
    \item \textbf{Assegnare arbitrariamente} i versi delle correnti
    \item \textbf{Applicare la convenzione degli utilizzatori/generatori} sui resistori: $+$ dove entra la corrente
    \item \textbf{Applicare KCL} ai nodi: $\sum I_{entranti} = \sum I_{uscenti}$
    \item \textbf{Applicare KVL} alle maglie: percorrere la maglia e sommare algebricamente le tensioni
    \item \textbf{Interpretare i risultati}: valori negativi indicano versi opposti
\end{enumerate}

\newpage
\section{Resistenze in Serie}

\subsection{Teoria}
Due o più resistenze sono collegate in serie quando sono attraversate dalla stessa corrente.

\begin{center}
\begin{circuitikz}[scale=1.2]
    \draw (0,0) to[battery1, v=$V$] (0,3)
          to[short] (1,3)
          to[R=$R_1$] (3,3)
          to[R=$R_2$] (5,3)
          to[R=$R_3$] (7,3)
          to[short] (7,0)
          to[short] (0,0);
    \draw[->, thick, red] (2,3.3) -- (3,3.3) node[midway, above] {$I$};
\end{circuitikz}
\end{center}

\textbf{Resistenza equivalente:}
\begin{equation}
R_{eq} = R_1 + R_2 + R_3 + \ldots + R_n
\end{equation}

\textbf{Proprietà:}
\begin{itemize}
    \item La corrente è la stessa in tutti i componenti: $I_{tot} = I_1 = I_2 = I_3$
    \item La tensione totale è la somma delle tensioni parziali: $V_{tot} = V_1 + V_2 + V_3$
    \item La resistenza equivalente è sempre maggiore della resistenza più grande
\end{itemize}

\subsection{Esercizi Svolti}

\textbf{Esercizio 2.1:} Calcolare la resistenza equivalente di tre resistori in serie: $R_1 = 100\,\Omega$, $R_2 = 220\,\Omega$, $R_3 = 330\,\Omega$.

\textit{Soluzione:}
\begin{align*}
R_{eq} &= R_1 + R_2 + R_3 = 100 + 220 + 330 = 650\,\Omega
\end{align*}

\textbf{Esercizio 2.2:} Un circuito serie è formato da tre resistori: $R_1 = 1\,k\Omega$, $R_2 = 2.2\,k\Omega$, $R_3 = 4.7\,k\Omega$. Il circuito è alimentato da una tensione di $24\,V$. Calcolare:
\begin{enumerate}
    \item La resistenza equivalente
    \item La corrente totale
    \item La tensione ai capi di ciascun resistore
\end{enumerate}

\textit{Soluzione:}
\begin{align*}
1)\quad R_{eq} &= 1000 + 2200 + 4700 = 7900\,\Omega = 7.9\,k\Omega \\
2)\quad I &= \frac{V_{tot}}{R_{eq}} = \frac{24\,V}{7900\,\Omega} = 3.04\,mA \\
3)\quad V_1 &= R_1 \cdot I = 1000 \cdot 0.00304 = 3.04\,V \\
V_2 &= R_2 \cdot I = 2200 \cdot 0.00304 = 6.69\,V \\
V_3 &= R_3 \cdot I = 4700 \cdot 0.00304 = 14.27\,V
\end{align*}

Verifica: $V_1 + V_2 + V_3 = 3.04 + 6.69 + 14.27 = 24\,V$ \checkmark

\newpage
\section{Resistenze in Parallelo}

\subsection{Teoria}
Due o più resistenze sono collegate in parallelo quando hanno gli stessi punti di collegamento, quindi la stessa tensione ai loro capi.
\begin{center}
\begin{circuitikz}[scale=1.2]
    \draw (0,0) to[battery1, v=$V$] (0,3)
          to[short] (2,3)
          to[R=$R_1$] (2,1.5)
          to[short] (2,0)
          to[short] (0,0);
    \draw (2,3) to[short] (4,3)
          to[R=$R_2$] (4,1.5)
          to[short] (4,0)
          to[short] (2,0);
    \draw (4,3) to[short] (6,3)
          to[R=$R_3$] (6,1.5)
          to[short] (6,0)
          to[short] (4,0);
    \draw[->, thick, red] (2.5,2.8) -- (2.5,2.2) node[midway, right] {$I_1$};
    \draw[->, thick, red] (4.5,2.8) -- (4.5,2.2) node[midway, right] {$I_2$};
    \draw[->, thick, red] (6.5,2.8) -- (6.5,2.2) node[midway, right] {$I_3$};
\end{circuitikz}
\end{center}

\textbf{Resistenza equivalente:}
\begin{equation}
\frac{1}{R_{eq}} = \frac{1}{R_1} + \frac{1}{R_2} + \frac{1}{R_3} + \ldots + \frac{1}{R_n}
\end{equation}

Per due resistenze in parallelo:
\begin{equation}
R_{eq} = \frac{R_1 \cdot R_2}{R_1 + R_2}
\end{equation}

\textbf{Proprietà:}
\begin{itemize}
    \item La tensione è la stessa su tutti i componenti: $V_{tot} = V_1 = V_2 = V_3$
    \item La corrente totale è la somma delle correnti parziali: $I_{tot} = I_1 + I_2 + I_3$
    \item La resistenza equivalente è sempre minore della resistenza più piccola
\end{itemize}

\subsection{Esercizi Svolti}

\textbf{Esercizio 3.1:} Calcolare la resistenza equivalente di due resistori in parallelo: $R_1 = 100\,\Omega$, $R_2 = 150\,\Omega$.

\textit{Soluzione:}
\begin{align*}
R_{eq} &= \frac{R_1 \cdot R_2}{R_1 + R_2} = \frac{100 \cdot 150}{100 + 150} = \frac{15000}{250} = 60\,\Omega
\end{align*}

\textbf{Esercizio 3.2:} Calcolare la resistenza equivalente di tre resistori in parallelo: $R_1 = 300\,\Omega$, $R_2 = 600\,\Omega$, $R_3 = 900\,\Omega$.

\textit{Soluzione:}
\begin{align*}
\frac{1}{R_{eq}} &= \frac{1}{300} + \frac{1}{600} + \frac{1}{900} \\
&= \frac{6 + 3 + 2}{1800} = \frac{11}{1800} \\
R_{eq} &= \frac{1800}{11} = 163.6\,\Omega
\end{align*}

\textbf{Esercizio 3.3:} Un circuito parallelo è formato da tre resistori: $R_1 = 1\,k\Omega$, $R_2 = 2\,k\Omega$, $R_3 = 4\,k\Omega$. Il circuito è alimentato da una tensione di $12\,V$. Calcolare:
\begin{enumerate}
    \item La resistenza equivalente
    \item La corrente in ciascun resistore
    \item La corrente totale
\end{enumerate}

\textit{Soluzione:}
\begin{align*}
1)\quad \frac{1}{R_{eq}} &= \frac{1}{1000} + \frac{1}{2000} + \frac{1}{4000} \\
&= \frac{4 + 2 + 1}{4000} = \frac{7}{4000} \\
R_{eq} &= \frac{4000}{7} = 571.4\,\Omega \\
2)\quad I_1 &= \frac{V}{R_1} = \frac{12}{1000} = 12\,mA \\
I_2 &= \frac{V}{R_2} = \frac{12}{2000} = 6\,mA \\
I_3 &= \frac{V}{R_3} = \frac{12}{4000} = 3\,mA \\
3)\quad I_{tot} &= I_1 + I_2 + I_3 = 12 + 6 + 3 = 21\,mA
\end{align*}

Verifica: $I_{tot} = \frac{V}{R_{eq}} = \frac{12}{571.4} = 21\,mA$ \checkmark

\newpage
\section{Partitore di Tensione e Partitore di Corrente}

\subsection{Partitore di Tensione}

\subsubsection{Teoria}

Il \textbf{partitore di tensione} è una configurazione fondamentale in elettrotecnica che permette di ottenere una tensione ridotta da una tensione di alimentazione maggiore, utilizzando resistenze in serie. Cerca di avere sempre corrente convezione dei utilizzatori quando applichi la formula.

\textbf{Configurazione:}

\begin{center}
\begin{circuitikz}[scale=1.5]
    \draw (0,0) to[battery1, invert, v=$V_{in}$] (0,4)
          to[short] (1,4)
          to[R=$R_1$, v<=$V_1$, i>^=$I$] (1,2)
          node[circ, label=right:$V_{out}$] {}
          to[R=$R_2$, v<=$V_2$, i>^=$I$] (1,0)
          to[short] (0,0);
    
    \node at (0.5,4) {$+$};
    \node at (0.5,0) {$-$};
    \node at (0.5,2) {$+$};
    \node at (1.5,2) {$-$};
\end{circuitikz}
\end{center}

\textbf{Formula del Partitore di Tensione:}

La tensione ai capi di una resistenza in un circuito serie è proporzionale al suo valore rispetto alla resistenza totale:

\begin{equation}
V_1 = V_{in} \cdot \frac{R_1}{R_1 + R_2}
\end{equation}

\begin{equation}
V_2 = V_{in} \cdot \frac{R_2}{R_1 + R_2} = V_{out}
\end{equation}

\textbf{Formula generale per $n$ resistenze:}

\begin{equation}
V_k = V_{in} \cdot \frac{R_k}{\sum_{i=1}^{n} R_i}
\end{equation}

\textbf{Proprietà:}
\begin{itemize}
    \item La somma delle tensioni parziali è uguale alla tensione totale: $V_1 + V_2 = V_{in}$
    \item La corrente è la stessa in tutte le resistenze
    \item Il partitore funziona solo a vuoto o con carichi ad alta impedenza
    \item La tensione su ciascuna resistenza è proporzionale al suo valore
\end{itemize}

\subsubsection{Esercizi Svolti}

\textbf{Esercizio 5.1:} Dato un partitore di tensione con $V_{in} = 12\,V$, $R_1 = 300\,\Omega$, $R_2 = 100\,\Omega$. Calcolare $V_{out}$ (tensione su $R_2$).

\begin{center}
\begin{circuitikz}[scale=1.3]
    \draw (0,0) to[battery1, invert, v=$12\,V$] (0,3)
          to[short] (1,3)
          to[R=$300\,\Omega$, v<=$V_1$, i>^=$I$] (1,1.5)
          node[circ, label=right:$V_{out}$] {}
          to[R=$100\,\Omega$, v<=$V_{out}$, i>^=$I$] (1,0)
          to[short] (0,0);
\end{circuitikz}
\end{center}

\textit{Soluzione:}
\begin{align*}
V_{out} &= V_{in} \cdot \frac{R_2}{R_1 + R_2} = 12 \cdot \frac{100}{300 + 100} = 12 \cdot \frac{100}{400} = 12 \cdot 0.25 = 3\,V \\
V_1 &= V_{in} \cdot \frac{R_1}{R_1 + R_2} = 12 \cdot \frac{300}{400} = 9\,V
\end{align*}

\textbf{Verifica:} $V_1 + V_{out} = 9 + 3 = 12\,V$ \checkmark

\textbf{Corrente nel circuito:}
\begin{align*}
I &= \frac{V_{in}}{R_1 + R_2} = \frac{12}{400} = 30\,mA
\end{align*}

\textbf{Esercizio 5.2:} Progettare un partitore di tensione per ottenere $V_{out} = 5\,V$ da $V_{in} = 15\,V$, con $R_2 = 1\,k\Omega$. Calcolare $R_1$.

\textit{Soluzione:}

Dalla formula del partitore:
\begin{align*}
V_{out} &= V_{in} \cdot \frac{R_2}{R_1 + R_2} \\
5 &= 15 \cdot \frac{1000}{R_1 + 1000} \\
\frac{5}{15} &= \frac{1000}{R_1 + 1000} \\
\frac{1}{3} &= \frac{1000}{R_1 + 1000} \\
R_1 + 1000 &= 3000 \\
R_1 &= 2000\,\Omega = 2\,k\Omega
\end{align*}

\textbf{Verifica:}
\begin{align*}
V_{out} &= 15 \cdot \frac{1000}{2000 + 1000} = 15 \cdot \frac{1000}{3000} = 15 \cdot \frac{1}{3} = 5\,V \quad \checkmark
\end{align*}

\textbf{Esercizio 5.3:} Dato un partitore con tre resistenze in serie: $R_1 = 100\,\Omega$, $R_2 = 200\,\Omega$, $R_3 = 300\,\Omega$, alimentato da $V_{in} = 24\,V$. Calcolare le tensioni su ciascuna resistenza.

\begin{center}
\begin{circuitikz}[scale=1.3]
    \draw (0,0) to[battery1, invert, v=$24\,V$] (0,4)
          to[short] (1,4)
          to[R=$100\,\Omega$, v<=$V_1$, i>^=$I$] (1,3)
          to[R=$200\,\Omega$, v<=$V_2$, i>^=$I$] (1,1.5)
          to[R=$300\,\Omega$, v<=$V_3$, i>^=$I$] (1,0)
          to[short] (0,0);
\end{circuitikz}
\end{center}

\textit{Soluzione:}
\begin{align*}
R_{tot} &= 100 + 200 + 300 = 600\,\Omega \\
V_1 &= 24 \cdot \frac{100}{600} = 24 \cdot \frac{1}{6} = 4\,V \\
V_2 &= 24 \cdot \frac{200}{600} = 24 \cdot \frac{1}{3} = 8\,V \\
V_3 &= 24 \cdot \frac{300}{600} = 24 \cdot \frac{1}{2} = 12\,V
\end{align*}

\textbf{Verifica:} $V_1 + V_2 + V_3 = 4 + 8 + 12 = 24\,V$ \checkmark

\subsection{Partitore di Corrente}

\subsubsection{Teoria}

Il \textbf{partitore di corrente} è una configurazione che permette di dividere una corrente tra più resistenze in parallelo, in modo inversamente proporzionale ai loro valori.

\textbf{Configurazione:}

\begin{center}
\begin{circuitikz}[scale=1.5]
    \draw (0,0) to[I, l=$I_{in}$, i>^=$I_{in}$] (0,3)
          to[short] (2,3);
    
    \draw (2,3) to[short, i=$I_1$] (2,2.7)
          to[R=$R_1$, v<=$V$, i>^=$I_1$] (2,0.3)
          to[short] (2,0);
    
    \draw (2,3) to[short, i=$I_2$] (5,3)
          to[R=$R_2$, v<=$V$, i>^=$I_2$] (5,0)
          to[short] (2,0);
    
    \draw (2,0) to[short] (0,0);
    
    \node[circ, label=above:A] at (2,3) {};
    \node[circ, label=below:B] at (2,0) {};
\end{circuitikz}
\end{center}

\textbf{Formula del Partitore di Corrente:}

La corrente che attraversa una resistenza in un circuito parallelo è inversamente proporzionale al suo valore:

\begin{equation}
I_1 = I_{in} \cdot \frac{R_2}{R_1 + R_2}
\end{equation}

\begin{equation}
I_2 = I_{in} \cdot \frac{R_1}{R_1 + R_2}
\end{equation}

\textbf{Nota importante:} La corrente maggiore scorre nella resistenza minore! Cerca di averre la corrente in ingresso entrante e le correnti dei rami in modalità uscenti

\textbf{Formula generale per $n$ resistenze:}

\begin{equation}
I_k = I_{in} \cdot \frac{R_{eq}}{R_k}
\end{equation}

dove $R_{eq}$ è la resistenza equivalente del parallelo.

\textbf{Proprietà:}
\begin{itemize}
    \item La somma delle correnti parziali è uguale alla corrente totale: $I_1 + I_2 = I_{in}$
    \item La tensione è la stessa su tutte le resistenze
    \item La corrente si distribuisce in modo inversamente proporzionale alle resistenze
    \item La resistenza più piccola conduce la corrente maggiore
\end{itemize}

\subsubsection{Esercizi Svolti}

\textbf{Esercizio 5.4:} Dato un partitore di corrente con $I_{in} = 120\,mA$, $R_1 = 200\,\Omega$, $R_2 = 400\,\Omega$. Calcolare $I_1$ e $I_2$.

\begin{center}
\begin{circuitikz}[scale=1.3]
    \draw (0,0) to[I, l=$120\,mA$, i>^=$I_{in}$] (0,3)
          to[short] (2,3);
    
    \draw (2,3) to[short, i=$I_1$] (2,2.7)
          to[R=$200\,\Omega$, v<=$V$, i>^=$I_1$] (2,0.3)
          to[short] (2,0);
    
    \draw (2,3) to[short, i=$I_2$] (5,3)
          to[R=$400\,\Omega$, v<=$V$, i>^=$I_2$] (5,0)
          to[short] (2,0);
    
    \draw (2,0) to[short] (0,0);
\end{circuitikz}
\end{center}

\textit{Soluzione:}
\begin{align*}
I_1 &= I_{in} \cdot \frac{R_2}{R_1 + R_2} = 120 \cdot \frac{400}{200 + 400} = 120 \cdot \frac{400}{600} = 120 \cdot \frac{2}{3} = 80\,mA \\
I_2 &= I_{in} \cdot \frac{R_1}{R_1 + R_2} = 120 \cdot \frac{200}{600} = 120 \cdot \frac{1}{3} = 40\,mA
\end{align*}

\textbf{Verifica:} $I_1 + I_2 = 80 + 40 = 120\,mA = I_{in}$ \checkmark

\textbf{Osservazione:} $R_1 < R_2$ quindi $I_1 > I_2$ (la corrente maggiore passa nella resistenza minore)

\textbf{Esercizio 5.5:} Dato un partitore di corrente con tre resistenze: $R_1 = 100\,\Omega$, $R_2 = 150\,\Omega$, $R_3 = 300\,\Omega$, alimentato da $I_{in} = 330\,mA$. Calcolare le correnti in ciascuna resistenza.

\begin{center}
\begin{circuitikz}[scale=1.3]
    \draw (0,0) to[I, l=$330\,mA$, i>^=$I_{in}$] (0,3)
          to[short] (2,3);
    
    \draw (2,3) to[short, i=$I_1$] (2,2.7)
          to[R=$100\,\Omega$, v<=$V$, i>^=$I_1$] (2,0.3)
          to[short] (2,0);
    
    \draw (2,3) to[short, i=$I_2$] (4,3)
          to[R=$150\,\Omega$, v<=$V$, i>^=$I_2$] (4,0)
          to[short] (2,0);
    
    \draw (4,3) to[short, i=$I_3$] (6,3)
          to[R=$300\,\Omega$, v<=$V$, i>^=$I_3$] (6,0)
          to[short] (4,0);
    
    \draw (2,0) to[short] (0,0);
\end{circuitikz}
\end{center}

\textit{Soluzione:}

Prima calcoliamo la resistenza equivalente:
\begin{align*}
\frac{1}{R_{eq}} &= \frac{1}{100} + \frac{1}{150} + \frac{1}{300} = \frac{6 + 4 + 2}{600} = \frac{12}{600} = \frac{1}{50} \\
R_{eq} &= 50\,\Omega
\end{align*}

Ora calcoliamo le correnti usando la formula generale:
\begin{align*}
I_1 &= I_{in} \cdot \frac{R_{eq}}{R_1} = 330 \cdot \frac{50}{100} = 330 \cdot 0.5 = 165\,mA \\
I_2 &= I_{in} \cdot \frac{R_{eq}}{R_2} = 330 \cdot \frac{50}{150} = 330 \cdot \frac{1}{3} = 110\,mA \\
I_3 &= I_{in} \cdot \frac{R_{eq}}{R_3} = 330 \cdot \frac{50}{300} = 330 \cdot \frac{1}{6} = 55\,mA
\end{align*}

\textbf{Verifica:} $I_1 + I_2 + I_3 = 165 + 110 + 55 = 330\,mA = I_{in}$ \checkmark

\textbf{Osservazione:} $R_1 < R_2 < R_3$ quindi $I_1 > I_2 > I_3$ (ordine inverso rispetto alle resistenze)

\textbf{Esercizio 5.6:} Un circuito ha $I_{in} = 60\,mA$, $R_1 = 1\,k\Omega$, $R_2 = 2\,k\Omega$. Calcolare:
\begin{enumerate}
    \item Le correnti $I_1$ e $I_2$
    \item La tensione comune $V$
    \item La potenza dissipata in ciascuna resistenza
\end{enumerate}

\textit{Soluzione:}
\begin{align*}
1)\quad I_1 &= 60 \cdot \frac{2000}{1000 + 2000} = 60 \cdot \frac{2}{3} = 40\,mA \\
I_2 &= 60 \cdot \frac{1000}{3000} = 60 \cdot \frac{1}{3} = 20\,mA \\
2)\quad R_{eq} &= \frac{1000 \cdot 2000}{3000} = \frac{2000000}{3000} = 666.7\,\Omega \\
V &= I_{in} \cdot R_{eq} = 0.06 \cdot 666.7 = 40\,V \\
&\text{(oppure: } V = I_1 \cdot R_1 = 0.04 \cdot 1000 = 40\,V \text{)} \\
3)\quad P_1 &= I_1^2 \cdot R_1 = (0.04)^2 \cdot 1000 = 1.6\,W \\
P_2 &= I_2^2 \cdot R_2 = (0.02)^2 \cdot 2000 = 0.8\,W
\end{align*}

\subsection{Confronto tra Partitore di Tensione e di Corrente}

\begin{center}
\begin{tabular}{|l|c|c|}
\hline
\textbf{Caratteristica} & \textbf{Partitore di Tensione} & \textbf{Partitore di Corrente} \\
\hline
Configurazione & Serie & Parallelo \\
\hline
Grandezza divisa & Tensione & Corrente \\
\hline
Grandezza comune & Corrente & Tensione \\
\hline
Proporzionalità & Diretta ($V_k \propto R_k$) & Inversa ($I_k \propto \frac{1}{R_k}$) \\
\hline
Formula & $V_k = V_{in} \cdot \frac{R_k}{R_{tot}}$ & $I_k = I_{in} \cdot \frac{R_{eq}}{R_k}$ \\
\hline
Resistenza maggiore & Riceve tensione maggiore & Riceve corrente minore \\
\hline
\end{tabular}
\end{center}

\textbf{Regola pratica:}
\begin{itemize}
    \item \textbf{Partitore di tensione:} La resistenza più grande "prende" più tensione
    \item \textbf{Partitore di corrente:} La resistenza più piccola "prende" più corrente
\end{itemize}

\newpage
\section{Circuiti Misti (Serie-Parallelo)}

\subsection{Teoria}
I circuiti misti contengono sia collegamenti in serie che in parallelo. Per risolverli:
\begin{enumerate}
    \item Identificare i gruppi di resistenze in serie o parallelo
    \item Calcolare le resistenze equivalenti parziali
    \item Procedere per semplificazioni successive fino ad ottenere $R_{eq}$ totale
\end{enumerate}

\subsection{Esercizi Svolti}

\textbf{Esercizio 4.1:} Dato il seguente circuito misto:

\begin{center}
\begin{circuitikz}[scale=1.3]
    \draw (0,0) to[battery1, invert, v=$V_g{=}24\,V$, i>^=$I_{tot}$] (0,3)
          to[short] (1,3)
          to[R=$R_1{=}100\,\Omega$, v<=$V_1$, i>^=$I_{tot}$] (3,3);
    
    % Nodo A
    \node[circ, label=above:A] at (3,3) {};
    
    % Parallelo R2 e R3
    \draw (3,3) to[short, i=$I_2$] (3,2.7)
          to[R=$R_2{=}200\,\Omega$, v<=$V_{23}$, i>^=$I_2$] (3,0.3)
          to[short] (3,0);
    
    \draw (3,3) to[short, i=$I_3$] (6,3)
          to[R=$R_3{=}300\,\Omega$, v<=$V_{23}$, i>^=$I_3$] (6,0)
          to[short] (3,0);
    
    % Nodo B
    \node[circ, label=below:B] at (3,0) {};
    
    \draw (3,0) to[short] (0,0);
\end{circuitikz}
\end{center}

Calcolare:
\begin{enumerate}
    \item La resistenza equivalente del parallelo $R_2 \parallel R_3$
    \item La resistenza equivalente totale
    \item La corrente totale $I_{tot}$
    \item La tensione $V_1$ e $V_{23}$
    \item Le correnti $I_2$ e $I_3$
\end{enumerate}

\textit{Soluzione:}

\begin{align*}
1)\quad R_{23} &= \frac{R_2 \cdot R_3}{R_2 + R_3} = \frac{200 \cdot 300}{200 + 300} = \frac{60000}{500} = 120\,\Omega \\
2)\quad R_{eq} &= R_1 + R_{23} = 100 + 120 = 220\,\Omega \\
3)\quad I_{tot} &= \frac{V_g}{R_{eq}} = \frac{24}{220} = 0.109\,A = 109\,mA \\
4)\quad V_1 &= R_1 \cdot I_{tot} = 100 \cdot 0.109 = 10.9\,V \\
V_{23} &= R_{23} \cdot I_{tot} = 120 \cdot 0.109 = 13.1\,V \\
5)\quad I_2 &= \frac{V_{23}}{R_2} = \frac{13.1}{200} = 65.5\,mA \\
I_3 &= \frac{V_{23}}{R_3} = \frac{13.1}{300} = 43.7\,mA
\end{align*}

\textbf{Verifiche:}
\begin{itemize}
    \item $V_1 + V_{23} = 10.9 + 13.1 = 24\,V$ \checkmark
    \item $I_2 + I_3 = 65.5 + 43.7 = 109.2\,mA \approx I_{tot}$ \checkmark
\end{itemize}

\textbf{Esercizio 4.2:} Dato il seguente circuito misto:

\begin{center}
\begin{circuitikz}[scale=1.3]
    \draw (0,0) to[battery1, invert, v=$V_g{=}30\,V$, i>^=$I_{tot}$] (0,4)
          to[short] (1,4);
    
    % Nodo A
    \node[circ, label=above:A] at (1,4) {};
    
    % Primo ramo: R1 in serie
    \draw (1,4) to[short, i=$I_1$] (1,3.7)
          to[R=$R_1{=}150\,\Omega$, v<=$V_1$, i>^=$I_1$] (1,2)
          to[short] (1,0);
    
    % Secondo ramo: R2 e R3 in serie
    \draw (1,4) to[short, i=$I_{23}$] (3.5,4)
          to[R=$R_2{=}100\,\Omega$, v<=$V_2$, i>^=$I_{23}$] (3.5,2.5)
          to[R=$R_3{=}200\,\Omega$, v<=$V_3$, i>^=$I_{23}$] (3.5,0)
          to[short] (1,0);
    
    % Nodo B
    \node[circ, label=below:B] at (1,0) {};
    
    \draw (1,0) to[short] (0,0);
\end{circuitikz}
\end{center}

Calcolare:
\begin{enumerate}
    \item La resistenza equivalente del ramo serie $R_2 + R_3$
    \item La resistenza equivalente totale
    \item La corrente totale e le correnti nei due rami
    \item Tutte le tensioni
\end{enumerate}

\textit{Soluzione:}

\begin{align*}
1)\quad R_{23} &= R_2 + R_3 = 100 + 200 = 300\,\Omega \\
2)\quad R_{eq} &= \frac{R_1 \cdot R_{23}}{R_1 + R_{23}} = \frac{150 \cdot 300}{150 + 300} = \frac{45000}{450} = 100\,\Omega \\
3)\quad I_{tot} &= \frac{V_g}{R_{eq}} = \frac{30}{100} = 0.3\,A = 300\,mA \\
I_1 &= \frac{V_g}{R_1} = \frac{30}{150} = 0.2\,A = 200\,mA \\
I_{23} &= \frac{V_g}{R_{23}} = \frac{30}{300} = 0.1\,A = 100\,mA \\
4)\quad V_1 &= V_g = 30\,V \quad \text{(resistenza in parallelo)} \\
V_2 &= R_2 \cdot I_{23} = 100 \cdot 0.1 = 10\,V \\
V_3 &= R_3 \cdot I_{23} = 200 \cdot 0.1 = 20\,V
\end{align*}

\textbf{Verifiche:}
\begin{itemize}
    \item $I_1 + I_{23} = 200 + 100 = 300\,mA = I_{tot}$ \checkmark
    \item $V_2 + V_3 = 10 + 20 = 30\,V = V_g$ \checkmark
\end{itemize}

\textbf{Esercizio 4.3:} Dato il seguente circuito misto più complesso:

\begin{center}
\begin{circuitikz}[scale=1.2]
    \draw (0,0) to[battery1, invert, v=$V_g{=}48\,V$, i>^=$I_{tot}$] (0,3)
      to[short] (1,3)
      to[R=$R_1{=}50\,\Omega$, v<=$V_1$, i>^=$I_{tot}$] (3,3);
    
    % Nodo A
    \node[circ, label=above:A] at (3,3) {};
    
    % Parallelo
    \draw (3,3) to[short, i=$I_2$] (3,2.7)
          to[R=$R_2{=}100\,\Omega$, v<=$V_{23}$, i>^=$I_2$] (3,0.3)
          to[short] (3,0);
    
    \draw (3,3) to[short, i=$I_3$] (7,3)
          to[R=$R_3{=}150\,\Omega$, v<=$V_{23}$, i>^=$I_3$] (7,0)
          to[short] (3,0);
    
    % Nodo B
    \node[circ, label=below:B] at (3,0) {};
    
    % R4 finale
    \draw (3,0) to[R=$R_4{=}30\,\Omega$, v<=$V_4$, i>^=$I_{tot}$] (1,0)
          to[short] (0,0);
\end{circuitikz}
\end{center}

Calcolare:
\begin{enumerate}
    \item La resistenza equivalente totale
    \item La corrente totale
    \item Tutte le tensioni e correnti del circuito
\end{enumerate}

\textit{Soluzione:}

\begin{align*}
1)\quad R_{23} &= \frac{R_2 \cdot R_3}{R_2 + R_3} = \frac{100 \cdot 150}{100 + 150} = \frac{15000}{250} = 60\,\Omega \\
R_{eq} &= R_1 + R_{23} + R_4 = 50 + 60 + 30 = 140\,\Omega \\
2)\quad I_{tot} &= \frac{V_g}{R_{eq}} = \frac{48}{140} = 0.343\,A = 343\,mA \\
3)\quad V_1 &= R_1 \cdot I_{tot} = 50 \cdot 0.343 = 17.15\,V \\
V_{23} &= R_{23} \cdot I_{tot} = 60 \cdot 0.343 = 20.58\,V \\
V_4 &= R_4 \cdot I_{tot} = 30 \cdot 0.343 = 10.29\,V \\
I_2 &= \frac{V_{23}}{R_2} = \frac{20.58}{100} = 205.8\,mA \\
I_3 &= \frac{V_{23}}{R_3} = \frac{20.58}{150} = 137.2\,mA
\end{align*}

\textbf{Verifiche:}
\begin{itemize}
    \item $V_1 + V_{23} + V_4 = 17.15 + 20.58 + 10.29 = 48.02\,V \approx V_g$ \checkmark
    \item $I_2 + I_3 = 205.8 + 137.2 = 343\,mA = I_{tot}$ \checkmark
\end{itemize}


\textbf{Esercizio 4.4:} Dato il seguente circuito misto più complesso trovare $I_4$: 

\begin{center}
\begin{circuitikz}[scale=1.3]

    % Generatore di corrente (invertito)
    \draw (0,0) to[isource, invert, l_=$I_g{=}2\,\mathrm{A}$, i>^=$I_{tot}$] (0,4)
          to[short] (1,4)
          to[R=$R_1{=}40\,\Omega$, v<=$V_1$, i>^=$I_{tot}$] (3.5,4);

    % Nodo A
    \node[circ, label=above:A] at (3.5,4) {};

    % Ramo sinistro
    \draw (3.5,4) to[short, i=$I_2$] (3.5,3.7)
          to[R=$R_2{=}80\,\Omega$, v<=$V_{R2}$, i>^=$I_2$] (3.5,2)
          to[short] (3.5,1.7)
          to[R=$R_4{=}60\,\Omega$, v<=$V_{R4}$, i>^=$I_4$] (3.5,0.3)
          to[short] (3.5,0);

    % Ramo destro
    \draw (3.5,4) to[short, i=$I_3$] (7,4)
          to[R=$R_3{=}120\,\Omega$, v<=$V_{R3}$, i>^=$I_3$] (7,2)
          to[short] (7,1.7)
          to[R=$R_5{=}90\,\Omega$, v<=$V_{R5}$, i>^=$I_5$] (7,0)
          to[short] (3.5,0);

    % Ponte centrale (cortocircuito)
    \draw (3.5,2) to[short, i=$I_6$] (7,2);

    % Nodo B
    \node[circ, label=below:B] at (3.5,0) {};

    % Resistenza di chiusura inferiore
    \draw (3.5,0) to[R=$R_7{=}50\,\Omega$, v<=$VR7$, i>^=$I_{tot}$] (1,0)
          to[short] (0,0);

\end{circuitikz}
\end{center}

\textit{Soluzione:}

Questo circuito può essere risolto notando che il ponte centrale crea un cortocircuito tra i nodi intermedi dei due rami paralleli.

\textbf{Analisi del circuito:}

Il cortocircuito tra i punti intermedi rende le resistenze $R_4$ e $R_5$ inutilizzabili nel percorso della corrente (la corrente preferisce il percorso a resistenza zero).

Quindi il circuito si semplifica a:
\begin{itemize}
    \item $R_1$ in serie con il parallelo di $R_2$ e $R_3$
    \item Il tutto in serie con $R_7$
\end{itemize}

\begin{align*}
R_{23} &= \frac{R_2 \cdot R_3}{R_2 + R_3} = \frac{80 \cdot 120}{80 + 120} = \frac{9600}{200} = 48\,\Omega \\
R_{eq} &= R_1 + R_{23} + R_7 = 40 + 48 + 50 = 138\,\Omega \\
V_{eq} &= I_g \cdot R_{eq} = 2 \cdot 138 = 276\,V \\
V_1 &= R_1 \cdot I_{tot} = 40 \cdot 2 = 80\,V \\
V_{23} &= R_{23} \cdot I_{tot} = 48 \cdot 2 = 96\,V \\
V_7 &= R_7 \cdot I_{tot} = 50 \cdot 2 = 100\,V
\end{align*}

Poiché il nodo intermedio sinistro e destro sono collegati da un filo (cortocircuito), sono allo stesso potenziale. Quindi:

\begin{align*}
I_2 &= \frac{V_{23}}{R_2} = \frac{96}{80} = 1.2\,A \\
I_3 &= \frac{V_{23}}{R_3} = \frac{96}{120} = 0.8\,A \\
I_6 &= I_2 - 0 = 1.2\,A \quad \text{(corrente nel ponte)}
\end{align*}

Dal momento che la corrente può passare attraverso il cortocircuito, tutta la corrente che arriva in $R_2$ passa per il cortocircuito e non attraversa $R_4$:

\begin{equation}
\boxed{I_4 = 0\,A}
\end{equation}

Allo stesso modo, $I_5 = 0\,A$.

\textbf{Verifiche:}
\begin{itemize}
    \item $V_1 + V_{23} + V_7 = 80 + 96 + 100 = 276\,V$ \checkmark
    \item Al nodo superiore del cortocircuito: $I_2 = I_6 + I_4 \Rightarrow 1.2 = 1.2 + 0$ \checkmark
    \item $I_2 + I_3 = 1.2 + 0.8 = 2\,A = I_{tot}$ \checkmark
\end{itemize}

\newpage
\section{Teoremi di Thevenin e Norton}

\subsection{Teoria}

I teoremi di Thevenin e Norton sono strumenti fondamentali per semplificare l'analisi dei circuiti elettrici. Permettono di sostituire una rete complessa di generatori e resistenze con un circuito equivalente molto più semplice.

\subsubsection{Teorema di Thevenin}

Il \textbf{Teorema di Thevenin} afferma che:

\textit{Qualsiasi rete lineare di generatori e resistenze, vista da due terminali A e B, può essere sostituita da un generatore di tensione ideale $V_{Th}$ in serie con una resistenza $R_{Th}$.}

\begin{center}
\begin{circuitikz}[scale=1.3]
    % Rete complessa (box)
    \draw[thick] (0,0) rectangle (3,2);
    \node at (1.5,1) {Rete};
    \node at (1.5,0.5) {complessa};
    \draw (3,1.5) to[short, -o] (4,1.5) node[right] {A};
    \draw (3,0.5) to[short, -o] (4,0.5) node[right] {B};
    
    % Freccia di equivalenza
    \draw[->, very thick] (4.5,1) -- (5.5,1);
    
    % Equivalente di Thevenin
    \draw (6,0.5) to[battery1, invert, v=$V_{Th}$] (6,1.5)
          to[short] (7,1.5)
          to[R=$R_{Th}$, -o] (9,1.5) node[right] {A};
    \draw (6,0.5) to[short, -o] (9,0.5) node[right] {B};
\end{circuitikz}
\end{center}

\textbf{Parametri dell'equivalente di Thevenin:}

\begin{itemize}
    \item \textbf{$V_{Th}$} (Tensione di Thevenin): È la tensione a vuoto tra i terminali A e B, cioè la tensione misurata quando nessun carico è collegato
    \item \textbf{$R_{Th}$} (Resistenza di Thevenin): È la resistenza vista dai terminali A e B quando tutti i generatori indipendenti sono spenti (generatori di tensione sostituiti da cortocircuiti, generatori di corrente sostituiti da circuiti aperti)
\end{itemize}

\textbf{Procedura per trovare l'equivalente di Thevenin:}

\begin{enumerate}
    \item \textbf{Identificare i terminali A e B} di interesse
    \item \textbf{Calcolare $V_{Th}$}: 
    \begin{itemize}
        \item Rimuovere il carico (se presente)
        \item Calcolare la tensione a vuoto tra A e B
        \item $V_{Th} = V_{AB}$ (a circuito aperto)
    \end{itemize}
    \item \textbf{Calcolare $R_{Th}$}:
    \begin{itemize}
        \item Spegnere tutti i generatori indipendenti (I generatori di tensione diventano cortocircuiti, i generatori di corrente diventano circuiti aperti)
        \item Calcolare la resistenza equivalente vista dai terminali A e B
        \item $R_{Th} = R_{eq}$ (con generatori spenti)
    \end{itemize}
\end{enumerate}

\subsubsection{Teorema di Norton}

Il \textbf{Teorema di Norton} afferma che:

\textit{Qualsiasi rete lineare di generatori e resistenze, vista da due terminali A e B, può essere sostituita da un generatore di corrente ideale $I_{N}$ in parallelo con una resistenza $R_{N}$.}

\begin{center}
\begin{circuitikz}[scale=1.3]
    % Rete complessa (box)
    \draw[thick] (0,0) rectangle (3,2);
    \node at (1.5,1) {Rete};
    \node at (1.5,0.5) {complessa};
    \draw (3,1.5) to[short, -o] (4,1.5) node[right] {A};
    \draw (3,0.5) to[short, -o] (4,0.5) node[right] {B};
    
    % Freccia di equivalenza
    \draw[->, very thick] (4.5,1) -- (5.5,1);
    
    % Equivalente di Norton
    \draw (6,0.5) to[I, l=$I_{N}$, i>^=$I_N$] (6,1.5)
          to[short, -o] (7,1.5) node[right] {A};
    \draw (6,1.5) to[short] (8,1.5)
          to[R=$R_{N}$] (8,0.5)
          to[short] (6,0.5)
          to[short, -o] (7,0.5) node[right] {B};
\end{circuitikz}
\end{center}

\textbf{Parametri dell'equivalente di Norton:}

\begin{itemize}
    \item \textbf{$I_{N}$} (Corrente di Norton): È la corrente di cortocircuito tra i terminali A e B, cioè la corrente che scorre quando A e B sono collegati direttamente
    \item \textbf{$R_{N}$} (Resistenza di Norton): È la stessa della resistenza di Thevenin, cioè $R_{N} = R_{Th}$
\end{itemize}

\textbf{Procedura per trovare l'equivalente di Norton:}

\begin{enumerate}
    \item \textbf{Identificare i terminali A e B} di interesse
    \item \textbf{Calcolare $I_{N}$}:
    \begin{itemize}
        \item Cortocircuitare i terminali A e B
        \item Calcolare la corrente che scorre nel cortocircuito
        \item $I_{N} = I_{cc}$ (corrente di cortocircuito)
    \end{itemize}
    \item \textbf{Calcolare $R_{N}$}:
    \begin{itemize}
        \item Procedere come per $R_{Th}$
        \item $R_{N} = R_{Th}$
    \end{itemize}
\end{enumerate}

\subsubsection{Conversione tra Thevenin e Norton}

Gli equivalenti di Thevenin e Norton sono \textbf{duali} e possono essere convertiti l'uno nell'altro:

\begin{equation}
I_{N} = \frac{V_{Th}}{R_{Th}}
\end{equation}

\begin{equation}
V_{Th} = I_{N} \cdot R_{N}
\end{equation}

\begin{equation}
R_{Th} = R_{N}
\end{equation}

\begin{center}
\begin{circuitikz}[scale=1.3]
    % Thevenin
    \draw (0,0) to[battery1, invert, v=$V_{Th}$] (0,2)
          to[short] (1,2)
          to[R=$R_{Th}$, -o] (3,2) node[right] {A};
    \draw (0,0) to[short, -o] (3,0) node[right] {B};
    
    % Freccia doppia
    \draw[<->, very thick] (3.5,1) -- (4.5,1);
    \node at (4,1.5) {$I_N = \frac{V_{Th}}{R_{Th}}$};
    
    % Norton
    \draw (5,0) to[I, l=$I_{N}$, i>^=$I_N$] (5,2)
          to[short, -o] (6,2) node[right] {A};
    \draw (5,2) to[short] (7,2)
          to[R=$R_{N}$] (7,0)
          to[short] (5,0)
          to[short, -o] (6,0) node[right] {B};
\end{circuitikz}
\end{center}

\subsubsection{Quando Usare Thevenin o Norton}

\textbf{Usa Thevenin quando:}
\begin{itemize}
    \item Il carico da collegare è prevalentemente resistivo
    \item È più facile calcolare la tensione a vuoto
    \item Si vogliono analizzare gli effetti di carichi variabili
\end{itemize}

\textbf{Usa Norton quando:}
\begin{itemize}
    \item Il carico da collegare è una resistenza bassa
    \item È più facile calcolare la corrente di cortocircuito
    \item Si lavora con generatori di corrente
\end{itemize}

\subsection{Esercizi Svolti}

\textbf{Esercizio 6.1:} Trovare l'equivalente di Thevenin del seguente circuito visto dai terminali A e B:

\begin{center}
\begin{circuitikz}[scale=1.3]
    \draw (0,0) to[battery1, invert, v=$V_g{=}24\,V$] (0,3)
          to[short] (1,3)
          to[R=$R_1{=}100\,\Omega$, i>^=$I$] (3,3)
          to[short] (3,2.5);
    
    \draw (3,2.5) to[R=$R_2{=}200\,\Omega$] (3,0.5)
          to[short] (3,0);
    
    \draw (3,2.5) to[short] (5,2.5)
          to[R=$R_3{=}300\,\Omega$] (5,0.5)
          to[short] (5,0);
    
    \draw (3,0) to[short] (5,0)
          to[short] (5,0)
          to[short] (0,0);
    
    % Terminali A e B
    \draw (3,2.5) to[short, -o] (6,2.5) node[right] {A};
    \draw (3,0) to[short, -o] (6,0) node[right] {B};
\end{circuitikz}
\end{center}

\textit{Soluzione:}

\textbf{Passo 1: Calcolare $V_{Th}$}

La tensione di Thevenin è la tensione a vuoto tra A e B. Il circuito è un partitore di tensione tra R1 e $R_2$ e $R_3$ in parallelo.

\begin{align*}
R_{23} &= \frac{R_2 \cdot R_3}{R_2 + R_3} = \frac{200 \cdot 300}{500} = 120\,\Omega \\
V_{Th} &= V_g \cdot \frac{R_{23}}{R_1 + R_{23}} = 24 \cdot \frac{120}{220} = 24 \cdot 0.545 = 13.09\,V
\end{align*}

\textbf{Passo 2: Calcolare $R_{Th}$}

Spegniamo il generatore (cortocircuito) e calcoliamo la resistenza vista da A e B:

\begin{center}
\begin{circuitikz}[scale=1.3]
    \draw (0,0) to[short] (0,3)
          to[short] (1,3)
          to[R=$R_1{=}100\,\Omega$] (3,3)
          to[short] (3,2.5);
    
    \draw (3,2.5) to[R=$R_2{=}200\,\Omega$] (3,0.5)
          to[short] (3,0);
    
    \draw (3,2.5) to[short] (5,2.5)
          to[R=$R_3{=}300\,\Omega$] (5,0.5)
          to[short] (5,0);
    
    \draw (3,0) to[short] (0,0);
    
    % Terminali A e B
    \draw (3,2.5) to[short, -o] (6,2.5) node[right] {A};
    \draw (3,0) to[short, -o] (6,0) node[right] {B};
\end{circuitikz}
\end{center}

\begin{align*}
R_{23} &= \frac{200 \cdot 300}{500} = 120\,\Omega \\
R_{Th} &= R_1 \parallel R_{23} = \frac{100 \cdot 120}{220} = 54.5\,\Omega
\end{align*}

\textbf{Equivalente di Thevenin:}

\begin{center}
\begin{circuitikz}[scale=1.3]
    \draw (0,0) to[battery1, invert, v=$V_{Th}{=}13.09\,V$] (0,2)
          to[short] (1,2)
          to[R=$R_{Th}{=}54.5\,\Omega$, -o] (3,2) node[right] {A};
    \draw (0,0) to[short, -o] (3,0) node[right] {B};
\end{circuitikz}
\end{center}

\textbf{Esercizio 6.2:} Per il circuito dell'esercizio precedente, trovare anche l'equivalente di Norton e verificare la conversione.

\textit{Soluzione:}

\textbf{Metodo 1: Da Thevenin a Norton}

\begin{align*}
I_{N} &= \frac{V_{Th}}{R_{Th}} = \frac{13.09}{54.5} = 0.240\,A = 240\,mA \\
R_{N} &= R_{Th} = 54.5\,\Omega
\end{align*}

\textbf{Metodo 2: Calcolo diretto}

Cortocircuitiamo A e B e calcoliamo la corrente:

Nel cortocircuito, tutta la corrente passa per il corto, quindi $R_3$ e $R_2$ sono in parallelo al cortocircuito (quindi ignorate).

\begin{align*}
I_{cc} &= \frac{V_g}{R_1} = \frac{24}{100} = 0.24\,A = 240\,mA
\end{align*}

\textbf{Equivalente di Norton:}

\begin{center}
\begin{circuitikz}[scale=1.3]
    \draw (0,0) to[I, l=$I_{N}{=}240\,mA$, i>^=$I_N$] (0,2)
          to[short, -o] (1,2) node[right] {A};
    \draw (0,2) to[short] (2,2)
          to[R=$R_{N}{=}54.5\,\Omega$] (2,0)
          to[short] (0,0)
          to[short, -o] (1,0) node[right] {B};
\end{circuitikz}
\end{center}

\textbf{Verifica:} $V_{Th} = I_N \cdot R_N = 0.24 \cdot 54.5 = 12.96 \approx 13.09\,V$ \checkmark

\textbf{Esercizio 6.3:} Trovare l'equivalente di Thevenin del seguente circuito visto dai terminali A e B:

\begin{center}
\begin{circuitikz}[scale=1.3]
    \draw (0,0) to[battery1, invert, v=$V_{g1}{=}12\,V$] (0,2)
          to[short] (1,2)
          to[R=$R_1{=}50\,\Omega$, i>^=$I_1$] (3,2)
          to[short] (3,1.5);
    
    \draw (3,1.5) to[R=$R_2{=}100\,\Omega$] (3,0.5)
          to[short] (3,0);
    
    \draw (3,0) to[short] (0,0);
    
    \draw (3,1.5) to[short] (5,1.5)
          to[R=$R_3{=}150\,\Omega$] (5,0)
          to[battery1, v<=$V_{g2}{=}6\,V$] (3,0);
    
    % Terminali A e B
    \draw (3,1.5) to[short, -o] (6,1.5) node[right] {A};
    \draw (3,0) to[short, -o] (6,0) node[right] {B};
\end{circuitikz}
\end{center}

\textbf{Esercizio 6.4:} Usando l'equivalente di Thevenin trovato nell'Esercizio 6.1, calcolare la corrente e la potenza dissipata in un carico $R_L = 100\,\Omega$ collegato tra A e B.

\textit{Soluzione:}

Colleghiamo il carico all'equivalente di Thevenin:

\begin{center}
\begin{circuitikz}[scale=1.3]
    \draw (0,0) to[battery1, invert, v=$V_{Th}{=}13.09\,V$] (0,2)
          to[short] (1,2)
          to[R=$R_{Th}{=}54.5\,\Omega$] (3,2)
          to[R=$R_L{=}100\,\Omega$, v<=$V_L$, i>^=$I_L$] (3,0)
          to[short] (0,0);
\end{circuitikz}
\end{center}

\begin{align*}
I_L &= \frac{V_{Th}}{R_{Th} + R_L} = \frac{13.09}{54.5 + 100} = \frac{13.09}{154.5} = 0.0847\,A = 84.7\,mA \\
V_L &= R_L \cdot I_L = 100 \cdot 0.0847 = 8.47\,V \\
P_L &= V_L \cdot I_L = 8.47 \cdot 0.0847 = 0.717\,W = 717\,mW
\end{align*}

Oppure: $P_L = R_L \cdot I_L^2 = 100 \cdot (0.0847)^2 = 0.717\,W$ \checkmark

\textbf{Esercizio 6.5:} Per il circuito dell'Esercizio 6.1, determinare il valore di $R_L$ che massimizza la potenza trasferita al carico (teorema del massimo trasferimento di potenza).

\textit{Soluzione:}

Il \textbf{teorema del massimo trasferimento di potenza} afferma che la massima potenza viene trasferita al carico quando:

\begin{equation}
R_L = R_{Th}
\end{equation}

Quindi:
\begin{align*}
R_L &= R_{Th} = 54.5\,\Omega
\end{align*}

La potenza massima trasferita è:
\begin{align*}
I_L &= \frac{V_{Th}}{2 R_{Th}} = \frac{13.09}{2 \cdot 54.5} = 0.120\,A = 120\,mA \\
P_{max} &= \frac{V_{Th}^2}{4 R_{Th}} = \frac{(13.09)^2}{4 \cdot 54.5} = \frac{171.35}{218} = 0.786\,W = 786\,mW
\end{align*}

\textbf{Osservazione:} Con $R_L = 100\,\Omega$ (Esercizio 6.4) abbiamo ottenuto $P_L = 717\,mW < P_{max}$.
\end{document}